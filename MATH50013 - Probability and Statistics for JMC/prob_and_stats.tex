\documentclass[12pt]{article}
\usepackage[a4paper, margin=1in]{geometry}
\usepackage{amsmath,amssymb,amsfonts,amsthm,mathtools,booktabs,cancel,multirow}

\newtheorem{thm}{Theorem}[section]
\newtheorem{lem}[thm]{Lemma}
\newtheorem{cor}[thm]{Corollary}
\newtheorem{prop}[thm]{Proposition}
\newtheorem*{prop*}{Proposition}

\theoremstyle{definition}
\newtheorem{defn}{Definition}[subsection]
\newtheorem*{defn*}{Definition}
\newtheorem*{not*}{Notation}

\DeclarePairedDelimiter\abs{\lvert}{\rvert}
\DeclarePairedDelimiter\length{\lVert}{\rVert}

\renewcommand{\d}{{\textnormal{d}}}
\newcommand{\R}{\mathbb{R}}
\newcommand{\N}{\mathbb{N}}
\newcommand{\F}{\mathcal{F}}
\newcommand{\X}{\mathbb{X}}
\newcommand{\Y}{\mathbb{Y}}
\newcommand{\RV}{\mathbf{RV}}

\newcommand{\mr}[1]{\multirow{2}{*}{#1}}

\DeclareMathOperator{\Var}{Var}
\DeclareMathOperator{\sd}{sd}
\DeclareMathOperator{\Bernoulli}{Bernoulli}
\DeclareMathOperator{\Binomial}{Binomial}
\DeclareMathOperator{\Geometric}{Geometric}
\DeclareMathOperator{\Poisson}{Poisson}
\DeclareMathOperator{\U}{U}
\DeclareMathOperator{\Uniform}{Uniform}
\DeclareMathOperator{\Exp}{Exp}
\DeclareMathOperator{\Cov}{Cov}
\DeclareMathOperator{\Cor}{Cor}
\DeclareMathOperator{\bias}{bias}
\DeclareMathOperator*{\argmax}{argmax}
\DeclareMathOperator{\Student}{Student}
\DeclareMathOperator{\freq}{freq}

\title{MATH50013 - Probability and Statistics for JMC}
\author{Notes by Robert Weingart}
\date{}

\begin{document}

\maketitle

\tableofcontents

\section{Introduction}

\subsection{Introduction to Uncertainty}

\subsection{Introduction to Statistics}

\subsubsection{Population vs. Sample}

\subsection{Probability AND Statistics}

\subsection{Statistical Modelling}

\section{Set Theory Review}

\subsection{Sets, subsets and complements}

\subsubsection{Sets}

\subsubsection{Membership, subsets, equality, complements, and singletons}

\subsection{Set operations}

\subsubsection{Venn diagrams, Unions and Intersections}

\subsubsection{Cartesian Products}

\subsection{Cardinality}

\section{Visual and Numerical Summaries}

\subsection{Visualization}

\subsubsection{The histogram}

\begin{defn*}
  A \textbf{histogram} partitions the range of a sample into \textbf{bins} and shows what number of data points in each bin.
  Rather than frequency, the amount shown can also be relative frequency or density.
\end{defn*}

\subsubsection{Empirical CDF}

\begin{defn*}
  The \textbf{indicator function} is defined as $I(\text{false}) := 0$ and $I(\text{true}) = 1$.
\end{defn*}

\begin{defn*}
  The \textbf{empirical cumulative distribution function} of a sample is
  $$F_n(x) := \frac{1}{n}\sum_{i = 1}^nI(x_i \leq x)$$
\end{defn*}

\subsection{Summary Statistics}

\subsubsection{Measures of Location}

\begin{defn*}
  The \textbf{arithmetic mean} is $\bar{x} := \frac{1}{n}\sum_{i = 1}^nx_i$.
\end{defn*}

\begin{defn*}
  The \textbf{geometric mean} is $x_G := \left(\prod_{i = 1}^nx_i\right)^{\frac{1}{n}}$.
\end{defn*}

\begin{defn*}
  The \textbf{harmonic mean} is $x_H := n\left(\sum_{i = 1}^n\frac{1}{x_i}\right)^{-1}$
\end{defn*}

\begin{defn*}
  The \textbf{$i$th order statistic}, written $x_{(i)}$, is the $i$th smallest value of the sample.
  For non-integer values of the form $i + \alpha$ with $\alpha \in (0, 1)$, we define
  $$x_{(i + \alpha)} := (1 - \alpha)x_{(i)} + \alpha x_{(i + 1)}$$
\end{defn*}

\begin{defn*}
  The \textbf{median} is $x_{(\frac{n + 1}{2})}$.
\end{defn*}

\begin{defn*}
  The \textbf{mode} is the most frequently occurring value.
  If there are multiple then the sample is \textbf{multimodal}.
\end{defn*}

\subsubsection{Measures of Dispersion}

\begin{defn*}
  The \textbf{mean square} or \textbf{sample variance} is
  $$s_x^2 := \frac{1}{n}\sum_{i = 1}^n(x_i - \bar{x})^2$$
\end{defn*}

\begin{defn*}
  The \textbf{root mean square} or \textbf{sample standard deviation} is
  $$s_x := \sqrt{\frac{1}{n}\sum_{i = 1}^n(x_i - \bar{x})^2}$$
\end{defn*}

\begin{defn*}
  The \textbf{range} is $x_{(n)} - x_{(1)}$.
\end{defn*}

\begin{defn*}
  The \textbf{first quartile} is $x_{\left(\frac{1}{4}(n + 1)\right)}$.
  The \textbf{third quartile} is $x_{\left(\frac{3}{4}(n + 1)\right)}$.
  The \textbf{interquartile range} is the difference between the third and first quartiles.
\end{defn*}

\subsubsection{Covariance and Correlation}

\begin{defn*}
  For a sample where each data point is an $(x_i, y_i)$ pair, the \textbf{covariance} is
  $$s_{xy} := \frac{1}{n}\sum_{i = 1}^n(x_i - \bar{x})(y_i - \bar{y}) = \frac{\sum_{i = 1}^nx_iy_i}{n} - \bar{x}\bar{y}$$.
\end{defn*}

\begin{defn*}
  For a sample as above, the \textbf{correlation} is
  $$r_{xy} := \frac{s_{xy}}{s_xs_y}$$
\end{defn*}

\subsubsection{Skewness}

\begin{defn*}
  The \textbf{skewness} is $\frac{1}{n}\sum_{i = 1}^n\left(\frac{x_i - \bar{x}}{s}\right)^3$.
\end{defn*}

\subsection{One more visualization: the box-and-whisker plot}

\begin{defn*}
  A \textbf{box-and-whisker plot} shows the median, first and third quartiles, points within $\frac{3}{2} \times IQR$ of the quartiles, and any outliers.
\end{defn*}

\section{Probability}

\subsection{The formal structure}

\subsubsection{$\sigma$-algebras}

\begin{defn}
  A \textbf{$\sigma$-algebra associated with $S$} is a set $\F$ of subsets of $S$ where $S \in \F$, $\F$ is closed under complements with respect to $S$, and $\F$ is closed under countable unions.
\end{defn}

\begin{prop*}
  $\emptyset \in \F$. $\F$ is also closed under countable intersections.
\end{prop*}

\subsubsection{Probability measure}

\begin{defn}
  A \textbf{probability measure} is a function $P : \F \to \R$ where $P(E) \geq 0$ for any $E$, $P(S) = 1$, and for countably many disjoint sets $E_i$,
  $$P\left(\bigcup_{i = 1}^{\infty}E_i\right) = \sum_{i = 1}^{\infty}P(E_i)$$.
  A triple $(S, \F, P)$ as previously defined is a \textbf{probability space}.
\end{defn}

\subsection{Interpretations of the probability space}

\subsection{Interpretation of the $\sigma$-algebra}

\subsubsection{The sample space ($S$)}

\begin{defn*}
  The \textbf{sample space} $S$ is the set of all possible outcomes of an experiment.
\end{defn*}

\subsubsection{The event space ($\F$)}

\begin{defn*}
  An \textbf{event} is a subset $E \subset S$.
  $\F$ is the set of all possible events being considered (which may not include all possible combinations of outcomes).
\end{defn*}

\begin{defn*}
  $E_1$ and $E_2$ are \textbf{mutually exclusive} iff $E_1 \cap E_2 = \emptyset$ i.e.\@ they cannot both happen at once.
\end{defn*}

\subsection{Interpretations of the probability measure ($P$)}

\subsubsection{Classical interpretation}

\begin{defn*}
  In the \textbf{classical interpretiation}, $S$ consists of finitely many equally likely \textbf{elementary events} and $P(E) = \frac{\abs{E}}{\abs{S}}$.
  For an infinite $S$, this can still be applied by replacing cardinality above with a different measure.
\end{defn*}

\subsubsection{Frequentist interpretation}

\begin{defn*}
  In the \textbf{frequentist interpretation}, when an experiment is repeated infinitely many times, the proportion of trials in which $E$ occurs approaches $P(E)$.
\end{defn*}

\subsubsection{Subjective interpretation}

\begin{defn*}
  In the \textbf{subjective interpretation}, $P(E)$ is the degree of belief a person has that $E$ occurs.
\end{defn*}

\subsection{A few derivations from the axioms}

\begin{prop*}
  For $E, F \in \F$,
  \begin{itemize}
    \item $P(\emptyset) = 0$
    \item $P(E) \leq 1$
    \item $P(\overline{E}) = 1 - P(E)$
    \item $P(E \cup F) = P(E) + P(F) - P(E \cap F)$
    \item $P(E \cap \overline{F}) = P(E) - P(E \cap F)$
    \item $E \subseteq F \implies P(E) \leq P(F)$
  \end{itemize}
\end{prop*}

\subsection{Conditional Probability}

\begin{defn}
  For $P(F) > 0$ the \textbf{conditional probability of $E$ given $F$} is
  $$P(E \mid F) := \frac{P(E \cap F)}{P(F)}$$
\end{defn}

\begin{prop*}
  For $P(F) > 0$,
  \begin{itemize}
    \item For any $E \in \F$, $P(E \mid F) \geq 0$
    \item $P(F \mid F) = 1$
    \item For $E_1, \ldots, E_n \in \F$ pairwise disjoint, $P\left(\bigcup_{i = 1}^nE_i \mid F\right) = \sum_{i = 1}^nP(E_i \mid F)$
  \end{itemize}
\end{prop*}

\subsection{Independent Events}

\begin{defn}
  $E, F \in \F$ are \textbf{independent} iff $P(E \cap F) = P(E)P(F)$.
  $E_1, \ldots E_n$ are \textbf{independent} iff for any subset $E_{i_1}, \ldots, E_{i_l}$ we have $P\left(\bigcap_{j = 1}^lE_{i_j}\right) = \prod_{j = 1}^lP(E_{i_j})$.
\end{defn}

\begin{prop*}
  $E$ and $F$ are independent $\implies E$ and $\overline{F}$ are independent.
\end{prop*}

\begin{prop*}
  $E$ and $F$ are independent $\iff P(E \mid F) = P(E)$.
\end{prop*}

\subsubsection{More Examples}

\subsubsection{Conditional Independence}

\begin{defn*}
  For $E_1, E_2, F \in \F$, $E_1$ and $E_2$ are \textbf{conditionally independent given $F$} iff $P(E_1 \cap E_2 \cap F) = P(E_1 \mid F)P(E_2 \mid F)$.
\end{defn*}

\subsubsection{Joint Events}

\begin{defn*}
  When combining multiple independent experiments, a \textbf{probability table} can be used to show the probabilities of all elementary events (i.e.\@ combinations of an elementary event in each experiment).
\end{defn*}

\subsection{Bayes's Theorem}

\setcounter{thm}{\arabic{subsection}} % why

\begin{thm}
  (Bayes's)
  For $E, F \in \F$ with $P(E) > 0$ and $P(F) > 0$,
  $$P(E \mid F) = \frac{P(F \mid E)P(E)}{P(F)}$$
\end{thm}

\begin{thm}
  (The Law of Total Probability)
  For a partition $E_1, \ldots$ of $S$, and any $F \in \F$, $P(F) = \sum_iP(F \mid E_i)P(E_i)$.
\end{thm}

\begin{thm}
  (Bayes's applied to a partition)
  For a partition $E_1, \ldots$ of $S$ with $P(E_i) > 0$ for all $i$ and $F \in \F$ with $P(F) > 0$,
  $$P(E_i \mid F) = \frac{P(F \mid E_i)P(E_i)}{\sum_jP(F \mid E_j)P(E_j)}$$
\end{thm}

\setcounter{subsection}{\arabic{thm}}

\subsection{More Examples}

\section{Discrete Random Variables}

\subsection{Random Variables}

\begin{defn}
  A \textbf{random variable} is a measurable mapping $X : S \to \R$ where $\forall x \in \R,\ \{s \in S : X(s) \leq x\} \in \F$.
\end{defn}

\begin{defn}
  The \textbf{range} of $X$ is $\X$, the image of $S$ under $X$.
\end{defn}

\begin{defn*}
  The \textbf{probability distribution} of $X$ is
  $$P_X(X \in B) := P(\{s \in S : X(S) \in B\})$$
  where $B \subseteq \R$.
\end{defn*}

\begin{not*}
  For brevity we write $\{X \in B\} := \{s \in S : X(s) \in B\}$ (TODO: doesn't this make $P$ and $P_X$ interchangeable?) and $\{a < X \leq b\} := \{X \in (a, b]\}$ etc.
\end{not*}

\subsubsection{Cumulative Distribution Function}

\begin{defn}
  The \textbf{cumulative distribution function} of $X$ is $F_X : \R \to [0, 1]$ where $F_X(x) = P_X(X \leq x)$.
\end{defn}

\begin{defn*}
  A function $f$ is \textbf{right-continuous} iff for any decreasing sequence $x_i \to x$ we have $f(x_i) \to f(x)$.
\end{defn*}

\begin{prop*}
  A CDF is right-continuous.
\end{prop*}

\begin{prop*}
  $F_X$ is a CDF iff all the following hold:
  \begin{itemize}
    \item $F_X$ is right-continuous
    \item $F_X(\R) \subseteq [0, 1]$
    \item $F_X$ is monotonically increasing
    \item $\lim_{x \to -\infty}F_X(x) = 0$
    \item $\lim_{x \to \infty}F_X(x) = 1$
  \end{itemize}
\end{prop*}

\subsection{Discrete Random Variables}

\begin{defn}
  A random variable is \textbf{discrete} iff its range is finite or countably infinite.
\end{defn}

\begin{defn}
  For a DRV $X$, the \textbf{probability mass function} $p_X : \R \to [0, 1]$ is $p_X(x) = P_X(X = x)$ for $x \in \X$ and $p_X(x) = 0$ for $x \notin \X$.
\end{defn}

\begin{defn*}
  The \textbf{support} of $X$ is $\{x \in \R : p_X(x) > 0\}$.
  Usually this is $\X$.
\end{defn*}

\subsubsection{Properties of Mass Function $p_X$}

\begin{prop*}
  An arbitrary function $p_X$ can be a PMF for $X$ iff $\forall x \in \X,\ p_X(x) \geq 0$ and $\sum_{x \in \X}p_X(x) = 1$.
\end{prop*}

\subsubsection{Discrete Cumulative Distribution Function}

\begin{defn*}
  The \textbf{cumulative distribution function} of a DRV $X$ is $F_X(x) = P(X \leq x)$ (TODO: is this not what it always is?).
\end{defn*}

\subsubsection{Connection between $F_X$ and $p_X$}

\begin{prop*}
  For $\X = \{x_1, \ldots\}$ with the $x_i \leq x_{i + 1}$ for all $i$,
  $$F_X(x) = \sum_{x_i \leq x}p_X(x_i)$$
  Equivalently,
  $$\forall i \geq 1,\ p_X(x_i) = F_X(x_i) - F_X(x_{i - 1})$$
\end{prop*}

\subsubsection{Properties of Discrete CDF $F_X$}

\begin{prop*}
  We have
  \begin{itemize}
    \item $\lim_{x \to -\infty}F_X(x) = 0$
    \item $\lim_{x \to \infty}F_X(x) = 1$
    \item $\lim_{h \to 0^+}F_X(x + h) = F_X(x)$
    \item $a < b \implies F_X(a) \leq F_X(b)$
    \item For $a < b$, $P(a < X \leq b) = F_X(b) - F_X(a)$
  \end{itemize}
\end{prop*}

\subsection{Functions of a discrete random variable}

\begin{prop*}
  For a DRV $X$ and $g : \X \to \R$, $Y = g(X)$ is also a DRV.
  We have
  $$p_Y(y) = \sum_{x \in \X : g(x) = y}p_X(x)$$
\end{prop*}

\subsection{Mean and Variance}

\begin{not*}
  All the functions defined in this section are of type $\RV \to \R$.
\end{not*}

\subsubsection{Expectation}

\begin{defn}
  The \textbf{expected value} or \textbf{mean} of a DRV $X$ is
  $$E_X(X) := \sum_{x \in \X}xp_X(x)$$
  It is often abbreviated to $E(X)$.
  For the case $E_Y(X)$ with $Y \neq X$, see below.
\end{defn}

\setcounter{thm}{\arabic{subsection}}

\begin{thm}
  For a \textbf{function of interest} $g : \R \to \R$, we have
  $$E_X(g(X)) = \sum_{x \in \X}g(x)p_X(x)$$
  This is the only situation where we can have $E_X(Y)$ with $X \neq Y$.
\end{thm}

\begin{prop*}
  $E$ is linear.
\end{prop*}

\stepcounter{subsection}

\begin{defn}
  For a DRV $X$, the \textbf{variance} of $X$ is
  $$\Var_X(X) := E_X\left((X - E_X(X))^2\right) = E(X^2) - E(X)^2$$
\end{defn}

\begin{prop*}
  For $a, b \in \R$, $\Var(aX + b) = a^2\Var(X)$
\end{prop*}

\begin{defn}
  For a DRV $X$, the \textbf{standard deviation} of $X$ is
  $$\sd(X) := \sqrt{\Var_X(X)}$$
\end{defn}

\begin{defn}
  For a DRV $X$, the \textbf{skewness} of $X$ is
  $$\gamma_1 := \frac{E_X((X - E_X(X))^3)}{\sd_X(X)^3}$$
\end{defn}

\subsubsection{Sums of Random Variables}

\begin{prop*}
  For $X_1, \ldots X_n$ (possibly with different distributions, not necessarily independent) with sum $S_n$, we have
  $$E(S_n) = \sum_{i = 1}^nE(X_i)$$
  and
  $$E\left(\frac{S_n}{n}\right) = \frac{1}{n}\sum_{i = 1}^nE(X_i)$$
\end{prop*}

\begin{prop*}
  For $X_1, \ldots X_n$ independent with sum $S_n$, we have
  $$\Var(S_n) = \sum_{i = 1}^n\Var(X_i)$$
  and
  $$\Var\left(\frac{S_n}{n}\right) = \frac{1}{n^2}\sum_{i = 1}^n\Var(X_i)$$
\end{prop*}

\begin{prop*}
  For $X_1, \ldots X_n$ independent and identically distributed with sum $S_n$,\\ $E(X_i) = \mu$ and $\Var(X_i) = \sigma^2$, we have
  $$E\left(\frac{S_n}{n}\right) = \mu$$ and $\Var\left(\frac{S_n}{n}\right) = \frac{\sigma^2}{n}$
\end{prop*}

\subsection{Some Important Discrete Random Variables}

\begin{tabular}{lccccc}
  \toprule
  $X$                           & $\X$               & $p_X(x)$                           & $E(X)$              & $\Var(X)$            & $\gamma_1$\\
  \midrule
  $X \sim \Bernoulli(p)$        & $\{0, 1\}$         & $p^x(1 - p)^{1 - x}$               & $p$                 & $p(1 - p)$           & $\frac{1 - 2p}{\sqrt{p(1 - p)}}\ast$ \\
  $X \sim \Binomial(n, p)$      & $\{0, \ldots n\}$  & $\binom{n}{x}p^x(1 - p)^{n - x}$   & $np$                & $np(1 - p)$          & $\frac{1 - 2p}{\sqrt{np(1 - p)}}$ \\
  $X \sim \Geometric(p)$        & $\{1, 2, \ldots\}$ & $p(1 - p)^{x - 1}$                 & $\frac{1}{p}$       & $\frac{1 - p}{p^2}$  & $\frac{2 - p}{\sqrt{1 - p}}$ \\
  $X \sim \Poisson(\lambda)$    & $\{0, 1, \ldots\}$ & $\frac{e^{-\lambda}\lambda^x}{x!}$ & $\lambda$           & $\lambda$            & $\frac{1}{\sqrt{\lambda}}$\\
  $X \sim \U(\{1, \ldots, n\})$ & $\{1, \ldots, n\}$ & $\frac{1}{n}$                      & $\frac{n + 1}{2}$   & $\frac{n^2 - 1}{12}$ & $0$ \\
  \bottomrule
\end{tabular}
\bigskip

$\ast$: The skewness of the Bernoulli distribution is not given in the official notes.

\subsubsection{Bernoulli Distribution}

$X \sim \Bernoulli(p)$ chooses between 1 and 0 where $P(X = 1) = p$.

\subsubsection{Binomial Distribution}

$X \sim \Binomial(n, p)$ is the total number of successes after $n$ Bernoulli trials with probability $p$.

\subsubsection{Geometric Distribution}

$X \sim \Geometric(p)$ is the number of Bernoulli trials with probability $p$ it will take to have the first success.

\subsubsection{Poisson Distribution}

$X \sim \Poisson(\lambda)$ is the number of occurrences of an event that occurs at a rate of $\lambda$.

\subsubsection{Discrete Uniform Distribution}

$X \sim \U(\{1, \ldots, n\})$ is a random value out of $\{1, \ldots n\}$.

\section{Continuous Random Variables}

\begin{defn}
  A random variable $X$ is absolutely \textbf{continuous} iff there exists a measurable non-negative function $f_X : \R \to \R$ (the \textbf{probability density function}) where
  $$\forall B \subseteq \R,\ P(X \in B) = \int_{x \in B}f_X(x)dx$$
\end{defn}

\subsubsection{Continuous Cumulative Distribution Function}

\begin{defn}
  The \textbf{cumulative distribution function} of a CRV $X$ is $F_X(x) = P(X \leq x)$ (as for any RV).
\end{defn}

\begin{prop*}
  For a CRV $X$, $F_X(x) = \int_{-\infty}^xf_X(x')dx'$
\end{prop*}

\subsubsection{Properties of Continuous $F_X$ and $f_X$}

\begin{prop*}
  For a CRV $X$,
  \begin{itemize}
    \item $\lim_{x \to -\infty}F_X(x) = 0$
    \item $\lim_{x \to \infty}F_X(x) = 1$
    \item If $F_X$ is differentiable at $x$ then $f_X(x) = F'_X(x)$
    \item $\forall a \in \R,\ P(X = a) = 0$
    \item For $a < b$, $P(a < X \leq b) = F_X(b) - F_X(a)$
    \item $f_X(X)$ is not a probability, so we do not require $f_X(x) \leq 1$
    \item $X$ is uniquely defined by $f_X$
  \end{itemize}
\end{prop*}

\begin{prop*}
  An arbitrary function $f_X$ is a PDF for a CRV iff $\forall x \in \R,\ f_X(x) \geq 0$ and $\int_{-\infty}^{\infty}f_X(x)dx = 1$ ($f_X$ is \textbf{normalised}).
\end{prop*}

\subsubsection{Transformations}

\begin{prop*}
  For $Y = g(X)$ with $g$ strictly monotonically increasing, we have
  $$F_Y(y) = F_X(g^{-1}(y))$$
  and
  $$f_Y(y) = f_X\left(g^{-1}(y)\right)g^{-1'}(y)$$
\end{prop*}

\begin{prop*}
  For $Y = g(X)$ with $g$ strictly monotonically decreasing, we have
  $$F_Y(y) = 1 - F_X(g^{-1}(y))$$
  and
  $$f_Y(y) = -f_X\left(g^{-1}(y)\right)g^{-1'}(y)$$
\end{prop*}

\subsection{Mean, Variance and Quantiles}

\subsubsection{Expectation}

\begin{defn}
  The \textbf{mean} or \textbf{expectation} of a CRV $X$ is
  $$E(X) := \int_{-\infty}^{\infty}xf_X(x)dx$$
\end{defn}

\begin{defn*}
  For any measurable \textbf{function of interest} $g : \R \to \R$ we have
  $$E(g(X)) := \int_{-\infty}^{\infty}g(x)f_X(x)dx$$
\end{defn*}

\begin{prop*}
  $E$ is linear.
\end{prop*}

\subsubsection{Variance}

\begin{defn}
  The \textbf{variance} of a CRV $X$ is
  $$\Var_X(X) = E((X - E(X))^2) = E(X^2) - E(X)^2$$
\end{defn}

\begin{prop*}
  For $a, b \in \R$, $\Var(aX + b) = a^2\Var(X)$
\end{prop*}

\subsubsection{Quantiles}

\begin{defn}
  For $\alpha \in [0, 1]$, we \textbf{$\alpha$-quantile} of a CRV $X$ is
  $$Q_X(\alpha) := F_X^{-1}(\alpha)$$
  so that $P(X \leq Q_X(\alpha)) = \alpha$.
\end{defn}

\subsection{Some Important Continuous Random Variables}

\begin{tabular}{lccccc}
  \toprule
  $X$                        & $\X$     & $f_X(x)$ & $F_X(x)$ & $E(X)$ & $\Var(X)$ \\
  \midrule
  $X \sim \U(a, b)$    & $(a, b)$ & $\begin{cases}\frac{1}{b - a} & a < x < b \\ 0 & \text{otherwise}\end{cases}$ & $\begin{cases}0 & x \leq a \\ \frac{x - a}{b - a} & a < x < b \\ 1 & x \geq b\end{cases}$ & $\frac{a + b}{2}$ & $\frac{(b - a)^2}{12}$\\
  $X \sim \Exp(\lambda)$     & $[0, \infty)$ & $\lambda e^{-\lambda x}$ & $1 - e^{-\lambda x}$ & $\frac{1}{\lambda}$ & $\frac{1}{\lambda^2}$ \\
  $X \sim \N(\mu, \sigma^2)$ & $\R$ & $\frac{1}{\sqrt{2\pi\sigma^2}}e^{-\frac{(x - \mu)^2}{2\sigma^2}}$ & $\frac{1}{\sqrt{2\pi\sigma^2}}\int_{-\infty}^xe^{-\frac{(t - \mu)^2}{2\sigma^2}}dt$ & $\mu$ & $\sigma^2$ \\
  \bottomrule
\end{tabular}

\subsubsection{Continuous Uniform Distribution}

$X \sim \U(a, b)$ or $X \sim \Uniform(a, b)$ is uniformly distributed on the interval $(a, b)$ and 0 elsewhere.

\begin{defn*}
  Tht \textbf{standard uniform} is $\Uniform(0, 1)$.
\end{defn*}

\begin{prop*}
  $X \sim \Uniform(0, 1) \implies (a + (b - a)X) \sim \Uniform(a, b)$.
\end{prop*}

\subsubsection{Exponential Distribution}

$X \sim \Exp(\lambda)$ is the time until an event occurring at rate $\lambda$ occurs.

\begin{prop*}
  $X \sim \Exp(\lambda)$ exhibits the \textbf{Lack of Memory Property}:
  $$\forall x, t > 0,\ P(X > t + x \mid X > t) = P(X > x)$$
\end{prop*}

\begin{prop*}
  If the number of events occurring in an interval of size $x$ is\\ $N_x \sim \Poisson(\lambda x)$ then the separation between two events is $X \sim \Exp(\lambda)$.
\end{prop*}

\subsubsection{Normal (Gaussian) Distribution}

$X \sim N(\mu, \sigma^2)$ has no obvious interpretation.

\begin{defn*}
  $X \sim N(0, 1)$ is the \textbf{standard normal distribution} or \textbf{unit normal distribution}.
  It has the PDF
  $$\phi(x) = \frac{1}{\sqrt{2\pi}}\exp\left(-\frac{1}{2}x^2\right)$$
  and the CDF
  $$\Phi(x) = \frac{1}{\sqrt{2\pi}}\int_{-\infty}^x\exp\left(-\frac{t^2}{2}\right)dt$$
\end{defn*}

\begin{prop*}
  $X \sim N(0, 1) \implies (\sigma X + \mu) \sim N(\mu, \sigma^2)$
\end{prop*}

\setcounter{thm}{\arabic{subsection}}

\begin{thm}
  (Central Limit Theorem)
  For $X_1, \ldots, X_n$ independent and identically distributed with mean $\mu$ and variance $\sigma^2$,
  $$\lim_{n \to \infty}\frac{\sum_{i = 1}^nX_i - n\mu}{\sqrt{n}\sigma} \sim N(0, 1)$$
\end{thm}

\stepcounter{subsection}

\subsection{Further examples}

\section{Joint Random Variables}

\begin{defn}
  For RVs $X$ and $Y$ with the same sample space, the \textbf{joint probability distribution} is $P_{XY}(B_X, B_Y) := P(X^{-1}(B_X) \cap Y^{-1}(B_Y))$ where $B_X, B_Y \subseteq \R$.
\end{defn}

\subsubsection{Joint Cumulative Distribution Function}

\begin{defn}
  The \textbf{joint cumulative distribution function} is\\ $F_{xy}(x, y) := P_{XY}(X \leq x, Y \leq y)$.
\end{defn}

\begin{prop*}
  $F_X(x) = F_{XY}(x, \infty)$ and $F_Y(y) = F_{XY}(\infty, y)$.
\end{prop*}

\subsubsection{Properties of Joint CDF $F_{XY}$}

\begin{prop*}
  And arbitrary function $F_{XY}$ is a valid joint CDF iff the following hold:
  \begin{itemize}
    \item $\forall x, y \in \R,\ F_{XY}(x, y) \in [0, 1]$
    \item $\forall x_1, x_2, y \in \R,\ x_1 < x_2 \implies F_{XY}(x_1, y) \leq F_{XY}(x_2, y)$
    \item $\forall x, y_1, y_2 \in \R,\ y_1 < y_2 \implies F_{XY}(x, y_1) \leq F_{XY}(x, y_2)$
    \item $\forall x, y \in \R$, $F_{XY}(x, -\infty) = F_{XY}(-\infty, y) = 0$
    \item $F_{XY}(\infty, \infty) = 1$
  \end{itemize}
\end{prop*}

\subsubsection{Joint Probability Mass Functions}

\begin{defn}
  For DRVs $X, Y$, the \textbf{joint probability mass function} is\\ $p_{XY}(x, y) := P_{XY}(X = x, Y = y)$.
\end{defn}

\begin{prop*}
  $p_X(x) = \sum_{y \in \Y}p_{XY}(x, y)$ and $p_Y(y) = \sum_{x \in \X}p_{XY}(x, y)$
\end{prop*}

\begin{prop*}
  An arbitrary function $p_{XY}$ is a valid joint PMF iff $\forall x, y \in \R, p_{XY}(x, y) \in [0, 1]$ and $\sum_{y \in \Y}\sum_{x \in \X}p_{XY}(x, y) = 1$.
\end{prop*}

\subsubsection{Joint Probability Density Functions}

\begin{defn*}
  CRVs $X$ and $Y$ are \textbf{jointly continuous} iff $\exists f_{XY} : \R \times \R \to \R$ where
  $$\forall B_{XY} \subseteq \R \times \R,\ P_{XY}(B_{XY}) = \int_{(x, y) \in B_{XY}}f_{XY}(x, y)dxdy$$
  Then $f_{XY}$ is the \textbf{joint probability density function} of $X$ and $Y$.
\end{defn*}

\begin{prop*}
  For jointly continuous CRVs, we have
  $$F_{XY}(x, y) = \int_{t = -\infty}^y\int_{s = -\infty}^xf_{XY}(s, t)dsdt$$
\end{prop*}

\begin{defn}
  (Not actually a definition)
  The joint PDF is
  $$f_{XY} = \frac{\partial^2}{\partial x\partial y}F_{XY}(x, y)$$
\end{defn}

\begin{prop*}
  $f_X(x) = \int_{y = -\infty}^{\infty}f_{XY}(x, y)dy$ and $f_Y(y) = \int_{x = -\infty}^{\infty}f_{XY}(x, y)dx$
\end{prop*}

\begin{prop*}
  An arbitrary function $f_{XY}$ is a valid joint PDF iff $\forall x, y \in \R,\ f_{XY}(x, y) \geq 0$ and $\int_{y = -\infty}^{\infty}\int_{x = -\infty}^{\infty}f_{XY}(x, y)dxdy = 1$.
\end{prop*}

\subsection{Independence, Conditional Probability, Expectation}

\subsubsection{Independence and conditional probability}

\begin{defn*}
  RVs $X$ and $Y$ are \textbf{independent} iff $\forall B_X, B_Y \subseteq \R,\ P_{XY}(B_X, B_Y) = P_X(B_X)P_Y(B_Y)$.
\end{defn*}

\begin{defn}
  CRVs $X$ and $Y$ are \textbf{independent} iff $\forall x, y \in \R,\ f_{XY}(x, y) = f_X(x)f_Y(y)$.
\end{defn}

\begin{defn}
  For RVs $X$ and $Y$, the \textbf{conditional probability distribution} is
  $$P_{Y \mid X}(B_Y \mid B_X) := \frac{P_{XY}(B_X, B_Y)}{P_X(B_X)}$$
\end{defn}

\begin{prop*}
  $X$ and $Y$ are independent $\iff \forall B_X, B_Y \subseteq \R,\ P_{Y \mid X}(B_Y \mid B_X) = P_Y(B_Y)$.
\end{prop*}

\begin{defn}
  For CRVs $X$ and $Y$, the \textbf{conditional probability density function} is
  $$f_{Y \mid X}(y \mid x) := \frac{f_{XY}(x, y)}{f_X(x)}$$
\end{defn}

\begin{prop*}
  $X$ and $Y$ are independent $\iff \forall x, y \in \R,\ f_{Y \mid X}(y \mid x) = f_Y(y)$.
\end{prop*}

\subsubsection{Expectation}

\begin{defn}
  For DRVs $X$ and $Y$:
  $$E_{XY}(g(X, Y)) := \sum_{y \in \Y}\sum_{x \in \X}g(x, y)p_{XY}(x, y)$$
\end{defn}

\begin{defn}
  For CRVs $X$ and $Y$:
  $$E_{XY}(g(X, Y)) := \int_{y = -\infty}^{\infty}\int_{x = -\infty}^{\infty}g(x, y)f_{XY}(x, y)dxdy$$
\end{defn}

\begin{prop*}
  Both versions of $E$ are linear.
\end{prop*}

\begin{prop*}
  $E_{XY}(g_1(X) + g_2(Y)) = E_X(g_1(X)) + E_Y(g_2(y))$.
  If $X$ and $Y$ are independent then $E_{XY}(g_1(x)g_2(y)) = E_X(g_1(x))E_Y(g_2(y))$.
\end{prop*}

\subsubsection{Conditional Expectiation}

\begin{defn}
  The \textbf{conditional expectation} of $Y$ given $X = x$ is
  $$E_{Y \mid X}(Y \mid X = x) := \sum_{y \in \Y}yp(y \mid x)$$
  or
  $$E_{Y \mid X}(Y \mid X = x) := \int_{y = -\infty}^{\infty}yf(y \mid x)dy$$
\end{defn}

\begin{defn*}
  The \textbf{covariance} of $X$ and $Y$ is
  $$\sigma_{XY} = \Cov(X, Y) := E_{XY}((X - E_X(X))(Y - E_Y(Y)))$$
\end{defn*}

\begin{defn}
  The \textbf{correlation} of $X$ and $Y$ is
  $$\rho_{XY} = \Cor(X, Y) := \frac{\sigma_{XY}}{\sigma_X\sigma_Y}$$
\end{defn}

\begin{prop*}
  $X$ and $Y$ are independent $\implies \sigma_{XY} = \rho_{XY} = 0$.
\end{prop*}

\subsection{Examples}

\subsection{Multivariate Transformations}

\subsubsection{Convolutions (sums of random variables)}

\setcounter{thm}{\arabic{subsection}}

\begin{thm}
  (Convolution Theorem)
  For independent RVs $X$ and $Y$ and $Z = X + Y$,
  $$p_Z(z) = \sum_{x \in \X}p_X(x)p_Y(z - x)$$
  or
  $$p_Z(z) = \int_{\R}f_X(x)f_Y(z - x)dx$$
\end{thm}

\stepcounter{subsection}

\begin{thm}
  If $X \sim N(\mu_X, \sigma_X^2)$ and $Y \sim N(\mu_Y, \sigma_Y^2)$ are independent then $X + Y \sim N(\mu_X + \mu_Y, \sigma_X^2 + \sigma_Y^2)$.
\end{thm}

\stepcounter{subsection}

\subsubsection{General Bivariate Transformations}

\begin{prop*}
  For DRVs $X$ and $Y$ with $U = g_1(X, Y)$ and $V = g_2(X, Y)$,
  $$p_{UV}(u, v) = \sum_{(x, y) \in A}p_{XY}(x, y)$$
  where
  $$A := \{(x, y) : (g_1(x, y), g_2(x, y)) = (u, v)\}$$
\end{prop*}

\begin{prop*}
  For CRVs $X$ and $Y$ with $U = g_1(X, Y)$ and $V = g_2(X, Y)$, and given $u := g_1(x, y)$ and $v := g_2(x, y)$,
  $$f_{UV}(u, v) = f_{XY}(x, y)\left\lvert\frac{\partial x}{\partial u}\frac{\partial y}{\partial v} - \frac{\partial x}{\partial v}\frac{\partial y}{\partial u}\right\rvert$$
  where
  $$A := \{(x, y) : (g_1(x, y), g_2(x, y)) = (u, v)\}$$
\end{prop*}

\begin{defn*}
  The \textbf{Gamma function} is $\Gamma(\alpha) = \int_0^{\infty}t^{\alpha - 1}e^{-t}dt$, defined for $\alpha \in (0, \infty)$.
\end{defn*}

\begin{prop*}
  We have:
  \begin{itemize}
    \item $\forall \alpha > 1,\ \Gamma(\alpha) = (\alpha - 1)\Gamma(\alpha)$
    \item $\Gamma(1) = 1$
    \item $\forall n \in \N,\ \Gamma(n) = (n - 1)!$
    \item $\Gamma\left(\frac{1}{2}\right) = \sqrt{\pi}$
  \end{itemize}
\end{prop*}

\begin{defn*}
  The \textbf{Gamma distribution} $X \sim \textnormal{Gamma}(\alpha, \beta)$ with $\alpha, \beta > 0$ has the following properties:
  \begin{itemize}
    \item $f_X(x) := \frac{\beta^{\alpha}}{\Gamma(\alpha)}x^{\alpha - 1}e^{-\beta x}$
    \item $\X = (0, \infty)$
    \item $E(X) = \frac{\alpha}{\beta}$
    \item $\Var(X) = \frac{\alpha}{\beta^2}$
  \end{itemize}
\end{defn*}

\begin{defn*}
  The \textbf{Beta function} is $B(\alpha, \beta) := \int_0^1x^{\alpha - 1}(1 - x)^{\beta - 1}dx = \frac{\Gamma(\alpha)\Gamma(\beta)}{\Gamma(\alpha + \beta)}$.
\end{defn*}

\begin{defn*}
  The \textbf{Beta distribution} $X \sim \textnormal{Beta}(\alpha, \beta)$ has PDF $f_X(x) := \frac{1}{B(\alpha, \beta)}x^{\alpha - 1}(1 - x)^{\beta - 1}$ and $\X = (0, 1)$
\end{defn*}

\setcounter{thm}{\arabic{subsection}}

\begin{thm}
  If $X \sim \textnormal{Gamma}(\lambda, \beta)$ and $Y \sim \textnormal{Gamma}(\xi, \beta)$ are independent then $X + Y \sim \textnormal{Gamma}(\lambda + \xi, \beta)$
\end{thm}

\section{Estimation}

\begin{not*}
  In this section we consider random variables which are known to have a distribution depending on an unknown parameter (so that $X \sim DIST(\theta)$ where $DIST$ is some distribution).
  $\Theta$ is the set of all possible values of $\theta$.
  For properties of $X$ which depend only on the distribution (essentially all of them), we use the notation $\mid \theta$ to indicate this dependence.
  For instance, we write $P_{X \mid \theta}(x \mid \theta)$ to mean whatever $P(X)$ would be if the missing parameter of the distribution were $\theta$.
  Note that this is entirely unrelated to all previous uses of the symbol $\mid$ in this document.
\end{not*}

\subsection{Estimators}

\begin{not*}
  Throughout this section, we consider a set of $n$ independent and identically distributed random variables $\underline{X} = (X_1, \ldots, X_n)$.
\end{not*}

\begin{defn}
  A \textbf{statistic} is a random variable $T$ which depends on $\underline{X}$.
  The corresponding lowercase letter $t : \R^n \to \R$ is used to represent a realised value of $T$.
\end{defn}

\begin{defn*}
  An \textbf{estimator} is a statistic used to compute unknown parameters $\theta$ of the distribution of $\underline{X}$.
  Its realised values are called \textbf{estimates}.
\end{defn*}

\subsubsection{Point estimates}

\begin{defn*}
  A \textbf{point estimate} is an estimator which estimates a single unknown parameter $\theta$.
  The official notes call this an estimate even though, according to the previous definition, it is an estimator rather than an estimate.
  The distribution of the point estimate, $P_{T \mid \theta}$, will depend on the same unknown parameter $\theta$.
\end{defn*}

\subsubsection{Bias, Efficiency, Consistency}

\begin{defn*}
  The \textbf{bias} of an estimator $T$ for a parameter $\theta$ is
  $$\bias(T, \theta) := E(T - \theta \mid \theta) = E(T \mid \theta) - \theta$$
\end{defn*}

\begin{defn*}
  $T$ is \textbf{unbiased} $\iff \forall \theta \in \Theta,\ \bias(T, \theta) = 0$.
\end{defn*}

\begin{prop*}
  For any distribution, the mean of a sample is an unbiased estimator for the mean of the distribution.
\end{prop*}

\begin{defn*}
  The \textbf{bias-corrected sample variance} of $\underline{X}$ is
  $$S_{n - 1}^2 := \frac{1}{n - 1}\sum_{i = 1}^n(X_i - \overline{X})^2$$
  This is an unbiased estimator for the variance of any distribution.
\end{defn*}

\begin{defn*}
  Given two unbiased estimators for the same parameter, $\widehat{\Theta}$ and $\widehat{\Psi}$, $\widehat{\Theta}$ is \textbf{more efficient} than $\widehat{\Psi}$ iff
  $$\left(\forall \theta \in \Theta,\ \Var\left(\widehat{\Theta} \mid \theta\right) \leq \Var\left(\widehat{\Psi} \mid \theta\right)\right) \land \left(\exists \theta \in \Theta : \Var\left(\widehat{\Theta} \mid \theta\right) < \Var\left(\widehat{\Psi} \mid \theta\right)\right)$$
  $\widehat{\Theta}$ is \textbf{efficient} iff it is more efficient than all other estimators.
\end{defn*}

\begin{defn*}
  $\widehat{\Theta}$ is \textbf{consistent} iff it converges in probability to $\theta$, that is to say
  $$\forall \theta \in \Theta,\ \forall \varepsilon > 0,\ \lim_{n \to \infty}P_{\widehat{\Theta} \mid \theta}\left(\left\lvert\left(\widehat{\Theta} \mid \theta\right) - \theta\right\rvert > \varepsilon\right) = 0$$
\end{defn*}

\begin{prop*}
  $\widehat{\Theta}$ is unbiased $\implies \widehat{\Theta}$ is consistent.
\end{prop*}

\subsubsection{Maximum Likelihood Estimation}

\begin{defn*}
  The \textbf{likelihood function} is
  $$L(\theta \mid \underline{x}) := \prod_{i = 1}^np_{X \mid \theta}(x_i)$$
  or
  $$L(\theta \mid \underline{x}) := \prod_{i = 1}^nf_{X \mid \theta}(x_i)$$
  where $\underline{x} = (x_1, \ldots, x_n)$ is a sample of $\underline{X}$.
  Note that this is yet another different usage of $\mid$.
\end{defn*}

\begin{defn*}
  The \textbf{maximum likelihood estimate} is $\widehat{\theta}_{MLE} := \argmax_{\theta \in \Theta}L(\theta \mid \underline{x})$.
\end{defn*}

\begin{defn*}
  The \textbf{log-likelihood function} is $\ell(\theta \mid \underline{x}) := \log{L(\theta \mid \underline{x})}$
\end{defn*}

\begin{defn*}
  The \textbf{maximum likelihood estimator} is defined like the maximum likelihood estimate and uses the same notation, but uses the RVs $\underline{X}$ instead of a specific sample $\underline{x}$.
\end{defn*}

\subsection{Confidence Intervals}

\subsubsection{Normal Distribution with Known Variance}

\begin{defn*}
  The \textbf{$(1 - \alpha)$ confidence interval} for the mean $\mu$ given a known variance $\sigma^2$ is
  $$\left[\overline{x} - z_{1 - \frac{\alpha}{2}}\frac{\sigma}{\sqrt{n}}, \overline{x} + z_{1 - \frac{\alpha}{2}}\frac{\sigma}{\sqrt{n}}\right]$$
  where $z_\alpha$ is the $\alpha$-quantile of $N(0, 1)$.
  Then a sample of size $n$ with this distribution should have $\overline{x}$ within this range $1 - \alpha$ of the time.
\end{defn*}

\subsubsection{Normal Distribution with Unknown Variance}

\begin{prop*}
  If $\mu$ and $\sigma^2$ are both unknown then 
  $$\frac{\overline{X} - \mu}{S_{n - 1} / \mu} \sim \Student(n - 1)$$
  where
  $$S_{n - 1} = \sqrt{\frac{\sum_{i = 1}^n(X_i - \overline{X})^2}{n - 1}}$$
  Then the $(1 - \alpha)$ confidence level for $\mu$ is
  $$\left[\overline{x} - t_{n - 1, 1 - \frac{\alpha}{2}}\frac{s_{n - 1}}{\sqrt{n}}, \overline{x} + t_{n - 1, 1 - \frac{\alpha}{2}}\frac{s_{n - 1}}{\sqrt{n}}\right]$$
  where $t_{\nu, \alpha}$ is the $\alpha$-quantile of $\Student(\nu)$.
\end{prop*}

\subsubsection{Another way to view the confidence interval: Neyman construction}

\begin{defn*}
  The \textbf{Neyman construction} is a graph with values of the estimator along the horizontal axis and values of the parameter along the vertical axis.
  For each value of the parameter, indicate a belt of values in which the estimator is expected to lie for that value.
  Draw a vertical line at the observed estimate.
  Then the range of parameter values whos belts intersect this lane is the confidence interval.
\end{defn*}

\section{Hypothesis Testing}

\begin{defn*}
  To test an unknown parameter $\theta \in \Theta$, partition $\Theta$ into $\Theta_0$ and $\Theta_1$.
  The \textbf{null hypothesis} is $H_0 : \theta \in \Theta_0$ (usually chosen to represent the absence of a finding - no correlation, the change has no effect on the data etc).
  The \textbf{alternative hypothesis} is $H_1 : \theta \in \Theta_1$.
  From now on, we use the $\mid$ notation from the last chapter with $H_0$ and $H_1$ instead of individual values of $\theta$.
\end{defn*}

\begin{defn*}
  To perform a hypothesis test, choose a \textbf{rejection region} $R \subseteq \R$ such that $P(T \in R \mid H_0) = \alpha$ is low.
  This $\alpha$ is the \textbf{significance level} of the test.
  Given an observed value $t(\underline{x})$, reject the null hypothesis iff $t \in R$.
  The \textbf{$p$-value} of a test is the significance level that lies on the boundary between rejecting and not rejecting the null hypothesis.
\end{defn*}

\subsubsection{Error Rates and Power of a Test}

\begin{defn*}
  \textbf{Type I error} occurs when $H_0$ is rejected, but is actually true.
  The probability of this is $\alpha$ by definition.
  \textbf{Type II error} occurs when $H_0$ is not rejected, but actually $H_1$ is true.
  Its probability is $\beta := P(T \notin R \mid \theta \in \Theta_1)$.
\end{defn*}

\begin{defn}
  The \textbf{power} of a hypothesis test is $1 - \beta = P(T \in R \mid \theta \in \Theta_1)$.
\end{defn}

\begin{defn*}
  A \textbf{simple hypothesis} or \textbf{point hypothesis} has the form $\theta = \theta_0$.
  A \textbf{composite hypothesis} has the form $\theta < \theta_0$ or $\theta > \theta_0$.
  A \textbf{two-sided test} tests a simple hypothesis against its negation.
  A \textbf{one-sided test} tests a composite hypothesis against its negation (usually the case $\theta = \theta_0$ is included in $H_0$).
\end{defn*}

\newpage
\begin{tabular}{c c c c c c}
  \toprule
  Dist(s)                 & $H_0$                & Test Stat & $R$ \\
  \midrule
  $N(\mu, \sigma^2)$      & $\mu = \mu_0$        & $\frac{\overline{X} - \mu_0}{\sigma / \sqrt{n}}$ & $\{z \mid \abs{z} > z_{1 - \frac{\alpha}{2}}\}$\\
  \addlinespace
  $N(\mu, \sigma^2)^\star$      & $\mu = \mu_0$        & $\frac{\overline{X} - \mu}{S_{n - 1} / \mu}$     & $\{t \mid \abs{t} > t_{n - 1, 1 - \frac{\alpha}{2}}\}$\\
  \addlinespace
  $N(\mu_X, \sigma^2_X),$ & \mr{$\mu_X = \mu_Y$} & \mr{$\frac{\overline{X} - \overline{Y}}{\sqrt{\frac{\sigma^2_X}{n_1} + \frac{\sigma^2_Y}{n_2}}}$} & \mr{$\{z \mid \abs{z} > z_{1 - \frac{\alpha}{2}}\}$} \\
  $N(\mu_Y, \sigma^2_Y)$  & & & \\
  \addlinespace
  $N(\mu_X, \sigma^2),$   & \mr{$\mu_X = \mu_Y$} & \mr{$\frac{\overline{X} - \overline{Y}}{S_{n_1 + n_2 - 2}\sqrt{\frac{1}{n_1} + \frac{1}{n_2}}}$} & \mr{$\{t \mid \abs{t} > t_{n_1 + n_2 - 2, 1 - \frac{\alpha}{2}}\}$} \\
  $N(\mu_Y, \sigma^2)^\star$    & & & \\
  \addlinespace
  $\theta \in \R^m, \abs{\X} = k$ & $\theta = \theta_0$  & $\sum\limits_{i = 1}^k\frac{(O_i - E_i)^2}{E_i}^\dag$ & $\{x^2 \mid x^2 > \chi^2_{k - m - 1,1-\alpha}\}$ \\
  \addlinespace
  $\abs{\X} = k,$          & $X$ and $Y$          & \mr{$\sum\limits_{j = 1}^\ell\sum\limits_{i = 1}^k\frac{(O_{ij} - E_{ij})^2}{E_{ij}}^\ddag$} & \mr{$\{x^2 \mid x^2 > \chi^2_{(k - 1)(\ell - 1),1-\alpha}\}$} \\
  $\abs{\Y} = \ell$        & independent      & & \\
  \addlinespace
  \bottomrule
\end{tabular}
\\

$\star$: $\sigma^2$ is not known.

$\dag$: $O_i = (\freq(\underline{X}, x_i), E_i = np_X(x_i)$

$\ddag$: $O_{ij} = \freq\left(\underline{(X, Y)}, (x_i, y_j)\right), E_{ij} = \freq(\underline{X}, x_i)\freq(\underline{Y}, y_j)$

\subsection{Testing for a population mean}

\subsubsection{Normal Distribution with Known Variance}

\begin{prop*}
  For $n$ variables i.i.d with distribution $N(\mu, \sigma^2)$ with $\mu$ unknown and $\sigma^2$ known and a hypothesis $H_0 : \mu = \mu_0$, use the test statistic
  $$Z = \frac{\overline{X} - \mu_0}{\sigma / \sqrt{n}} \sim \Phi$$
  and the rejection region $R = (-\infty, -z_{1 - \frac{\alpha}{2}}) \cup (z_{1 - \frac{\alpha}{2}}, \infty) = \{z \mid \abs{z} > z_{1 - \frac{\alpha}{2}}\}$ has significance level $\alpha$.
  The $p$-value is $2(1 - \Phi(\abs{z}))$.
\end{prop*}

\subsubsection{Normal Distribution with Unknown Variance}

\begin{prop*}
  If $\sigma^2$ is also unknown then we use
  $$T = \frac{\overline{X} - \mu}{S_{n - 1} / \mu} \sim \Student(n - 1)$$
  and the rejection region $R = (-\infty, -t_{n - 1, 1 - \frac{\alpha}{2}}) \cup (t_{n - 1, 1 - \frac{\alpha}{2}}, \infty) = \{t \mid \abs{t} > t_{n - 1, 1 - \frac{\alpha}{2}}\}$.
\end{prop*}

\begin{prop*}
  $\chi^2(k)$ has mean $k$ and variance $2k$.
\end{prop*}

\subsection{Testing for differences in population means}

\subsubsection{Two Sample Problems}

\begin{not*}
  In this section we have a sample $\underline{X} = (X_1, \ldots, X_{n_1})$ i. i. d with distribution $P_X$ and another sample $\underline{Y} = (Y_1, \ldots, Y_{n_2})$ i. i. d with distribution $P_Y$ with $\mu_X$ and $\mu_Y$ unknown and the samples independent from each other.
  In the special case where $\underline{X}$ and $\underline{Y}$ are \textbf{paired}, $n_1 = n_2$ and each pair $(X_i, Y_i)$ is possibly dependent.
\end{not*}

\subsubsection{Normal Distributions with Known Variances}

\begin{prop*}
  If $P_X = N(\mu_X, \sigma^2_X)$ with $\mu_X$ unknown and $P_Y = N(\mu_Y, \sigma^2_Y)$ with $\mu_Y$ unknown, we can test $H_0 : \mu_X = \mu_Y$ using the test statistic
  $$Z = \frac{\overline{X} - \overline{Y}}{\sqrt{\frac{\sigma^2_X}{n_1} + \frac{\sigma^2_Y}{n_2}}} \sim \Phi$$
  and the rejection region $R = \{z \mid \abs{z} > z_{1 - \frac{\alpha}{2}}\}$.
\end{prop*}

\subsubsection{Normal Distributions with Unknown Variances}

\begin{defn*}
  If $\sigma^2_X = \sigma^2_Y = \sigma^2$ unknown, the \textbf{bias-corrected pooled sample variance} is
  $$\frac{\sum_{i = 1}^{n_1}(X_i - \overline{X})^2 + \sum_{i = 1}^{n_2}(Y_i - \overline{Y})^2}{n_1 + n_2 - 2}$$
  This is an unbiased estimator for $\sigma^2$.
\end{defn*}

\begin{prop*}
  In the situation above, test $H_0 : \mu_X = \mu_Y$ using
  $$T = \frac{\overline{X} - \overline{Y}}{S_{n_1 + n_2 - 2}\sqrt{\frac{1}{n_1} + \frac{1}{n_2}}} \sim \Student(n_1 + n_2 - 2)$$
  and $R = \{t \mid \abs{t} > t_{n_1 + n_2 - 2, 1 - \frac{\alpha}{2}}\}$
\end{prop*}

\subsection{Goodness of Fit}

\subsubsection{Count Data and Chi-Square Tests}

\begin{not*}
  In this section, we have $\underline{X}$ with range $\{x_1, \ldots x_k\}$ and pmf $p_X(j) = P(X = x_j \mid \theta)$ where $\theta \in \R^m$ (there are $m$ unknown parameters which we express as an $m$-vector).
  We write $p_j \equiv p_X(j)$ for brevity.
  Consider the observed frequencies $\underline{O} = (O_1, \ldots, O_k)$ with $O_j = \abs{\{i \in \{1, \ldots, n\} \mid X_i = x_j\}}$
  We will not be referring to $\underline{X}$ directly after this, so just ignore the fact that $x_i$ can we can't express extracting values from a sample $\underline{x}$ right now because the notation is the same as the values in the range.
  For a hypothesis $H_0 : \theta = \theta_0$, we can compute the expected probabilites $\{p_1, \ldots, p_k\}$ and thus the expected frequencies $\underline{E} = (E_1, \ldots E_k)$ with $E_j = np_j$.
\end{not*}

\begin{prop*}
  In this situation, test $H_0 : \theta = \theta^n$ using the \textbf{chi-squared statistic}
  $$\chi^2 = \sum_{i = 1}^k\frac{(O_i - E_i)^2}{E_i} \sim \chi^2_{k - m - 1}$$
  and $R = \{x^2 \mid x^2 > \chi^2_{k - m - 1,1-\alpha}\}$
\end{prop*}

\subsubsection{Proportions}

\subsubsection{Model Checking}

\subsubsection{Independence}

\begin{prop*}
  For two DRVs $X$ and $Y$ with ranges $\{x_1, \ldots x_k\}$ and $\{y_1, \ldots, y_\ell\}$, test $H_0 : X$ and $Y$ are independent by creating a \textbf{contingency table}:\\
  \begin{tabular}{cccccc}
             & $y_1$           & $y_2$           & $\ldots$ & $y_\ell$          &\\
    $x_1$    & $n_{11}$        & $n_{12}$        & $\ldots$ & $n_{1\ell}$       & $n_{1\bullet}$\\
    $x_2$    & $n_{21}$        & $n_{22}$        & $\ldots$ & $n_{2\ell}$       & $n_{2\bullet}$\\
    $\vdots$ & $\vdots$        & $\vdots$        & $\ddots$ & $\vdots$          & $\ldots$\\
    $x_k$    & $n_{k1}$        & $n_{k2}$        & $\ldots$ & $n_{k\ell}$       & $n_{k\bullet}$\\
             & $n_{\bullet 1}$ & $n_{\bullet 2}$ & $\ldots$ & $n_{\bullet\ell}$ & $n$
  \end{tabular}\\
  where the $n_{ij}$ are the observed frequencies and the numbers at the end of each row and column are the sums of those frequencies.
  Let the expected frequency at each $i, j$ be $\frac{n_{i\bullet}n_{\bullet j}}{n}$ and perform a chi-squared test on these observations and expectations.
  Use the rejection region $\{x^2 \mid x^2 > \chi^2_{(k - 1)(\ell - 1),1-\alpha}\}$.
\end{prop*}

\subsubsection{The $\chi^2$ distribution and degrees of freedom}

\begin{prop*}
  The parameter of the $\chi^2$ distribution represents the size of the sample minus the number of ways in which the expectations depend on the sample.
\end{prop*}

\section{Convergence Concepts}

\end{document}
