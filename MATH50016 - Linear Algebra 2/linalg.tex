\documentclass[12pt]{article}
\usepackage[a4paper, margin=1in]{geometry}
\usepackage{amsmath,amssymb,amsfonts,amsthm,mathtools}

\newtheorem{thm}{Theorem}[section]
\newtheorem{lem}[thm]{Lemma}
\newtheorem{cor}[thm]{Corollary}
\newtheorem{prop}[thm]{Proposition}
\newtheorem*{prop*}{Proposition}

\theoremstyle{definition}
\newtheorem{defn}{Definition}[section]
\newtheorem*{defn*}{Definition}
\newtheorem*{not*}{Notation}

\DeclarePairedDelimiter\length{\lvert\lvert}{\rvert\rvert}
\DeclareMathOperator{\rank}{rank}
\DeclareMathOperator{\lcm}{lcm}
\DeclareMathOperator{\spn}{span}
\DeclareMathOperator{\chr}{char}
\DeclareMathOperator{\diag}{diag}

\title{MATH50003 Linear Algebra and Numerical Analysis Term 1\\{\Large also known as}\\MATH50016 Linear Algebra 2 for JMC}
\author{Notes by Robert Weingart}
\date{}

\begin{document}

\maketitle

\tableofcontents

\section*{General notes}

%This document is a concise summary of the official course notes.
%It includes every theorem, lemma, proposition, corollary and defintion, without any proofs or exercises.
%The numbering should match the official notes, but notation has been changed in many places to be more consistent.\\
%
%\noindent Notation which is used throughout an entire section is defined only once at the start rather than in every statement.
%When a symbol has the same meaning in multiple consecutive statements, the definition is not repeated as it should be clear from the context.\\
%
%\noindent Unnumbered definitions correspond to bolded terms that are not given numbers.
%Unnumbered propositions are results from exercises that I though were important enough to include.

\begin{not*}
	Throughout this entire document, unless otherwise stated:
	\begin{itemize}
		\item $F$ is an arbitrary field
		\item $V$ is an arbitrary vector space over $F$
		\item $T : V \to V$ is an arbitrary linear map
		\item $A$ is an arbitrary $n \times n$ matrix with entries in $F$
	\end{itemize}
	Furthermore,
	\begin{itemize}
		\item For any linear map or matrix $S$, $c_S$ is the characteristic polynomial
		\item For any linear map or matrix $S$, $m_S$ is the minimal polynomial (defined in section 9)
		\item To avoid ambiguity, $:$ always means ``such that'' and $\mid$ always means ``divides''
	\end{itemize}
\end{not*}

\section{Course Overview}

\noindent Nothing to summarise here, I am including the section to maintain correct numbering.

\section{Some revision from 1st Year Linear Algebra}

\begin{defn*}
	Let $V$ be finite-dimensional with a basis $B = \{v_1, \ldots, v_n\}$.
	Let $a_{ij}$ be the $j$th component of $T(v_i)$ with respect to $B$.
	The \textbf{matrix of $T$} with respect to $B$ is $[T]_B := (a_{ij})$.
\end{defn*}

\begin{prop}
	For linear maps $S$ and $T$ and a basis $B$, $[ST]_B = [S]_B[T]_B$.
\end{prop}

\begin{prop*}
	For any polynomial $q(x)$, $[q(T)]_B = q([T]_B)$.
\end{prop*}

\begin{defn*}
	Let $V$ be finite-dimensional with two bases $E = \{e_1, \ldots, e_n\}$ and $F = \{f_1, \ldots, f_n\}$.
	Let $p_{ij}$ be the $i$th component of $f_j$ with respect to $E$.
	The \textbf{change of basis matrix} from $E$ to $F$ is $(p_{ij})$.
\end{defn*}

\begin{prop}
	The change of basis matrix is invertible.
	If $P$ is the chnage of basis matrix from $E$ to $F$, then $[T]_F = P^{-1}[T]_EP$.
\end{prop}

\begin{defn*}
	The \textbf{characteristic polynomial} of $T$ is $c_T(x) := \det{xI_V - T}$.
\end{defn*}

\begin{prop}
	The eigenvalues of $T$ are the roots of $c_T$.
	The eigenvectors to an eigenvalue $\lambda$ are the non-zero vectors in $E_{\lambda} := \ker{\lambda I_V - T}$.
	$[T]_B$ is diagonal $\iff B$ consists of eigenvalues of $T$.
\end{prop}

\begin{defn*}
	The \textbf{$\lambda$-eigenspace} of $T$ is $E_{\lambda}$ as defined above.
\end{defn*}

\begin{prop}
	If $F = \mathbb{C}$ and $V$ is finite-dimensional then $T$ has an eigenvalue in $\mathbb{C}$.
\end{prop}

\begin{prop}
	Eigenvectors to different eigenvalues are linearly independent.
\end{prop}

\begin{cor}
	If the characteristic polynomial of $T$ has $\dim{V}$ distinct roots then $T$ is diagonalisable.
\end{cor}

\section{Algebraic and geometric multiplicities of eigenvalues}

\begin{defn*}
	If $\lambda$ is an eigenvalue of $T$, its \textbf{algebraic multiplicity} is $a(\lambda)$ such that $c_T(x) = (x - \lambda)^{a(\lambda)}q(x)$ where $q(\lambda) \neq 0$.
	Its \textbf{geometric multiplicity} is $g(\lambda) = \dim{E_{\lambda}}$.
\end{defn*}

\begin{prop}
	$g(\lambda) \leq a(\lambda)$.
\end{prop}

\begin{thm}
	If $\dim{V} = n$ and $\lambda_1, \ldots, \lambda_r$ are the distinct eigenvalues of $T : V \to V$, the following statements are equivalent:
	\begin{itemize}
		\item $T$ is diagonalisable
		\item $\sum_{i = 1}^rg(\lambda_i) = n$
		\item $\forall i,\ g(\lambda_i) = a(\lambda_i)$
	\end{itemize}
\end{thm}

\section{Direct sums}

\begin{defn*}
	$V$ is the \textbf{direct sum} of subspaces $V_1, \ldots, V_k$ iff
	$$\forall v \in V,\ \exists! v_1 \in V_1, \ldots, v_k \in V_k : v = \sum_{i = 1}^kv_i$$
	that is to say, the selection of $v_i$ is unique.
	In this case we write $V = V_1 \oplus \ldots \oplus V_k$.
\end{defn*}

\begin{not*}
	From now on I will use $\bigoplus$ to write a direct sum over a range, like $\sum$ for sums.
	This does not occur in the official notes, but I will use it here for brevity.
	For instance, in the situation above I will now write
	$$V = \bigoplus_{i = 1}^kV_i$$
\end{not*}

\begin{prop}
	$V = V_1 \oplus V_2 \iff V_1 \cap V_2 = \{0\}$ and $\dim{V_1} + \dim{V_2} = \dim{V}$
\end{prop}

\begin{prop}
	$V = \bigoplus_{i = 1}^kV_i \iff \dim{V} = \sum_{i = 1}^k\dim{V_i}$ and for any bases $B_i$ of the $V_i$, $\bigcup_{i = 1}^kB_i$ is a basis of $V$.
\end{prop}

\begin{defn}
	A subspace $W$ of $V$ is \textbf{$T$-invariant} $\iff T(W) \subseteq W$.
	In this case, the \textbf{restriction} of $T$ to $W$ is $T_W : W \to W$, where $T_W(w) = T(w)$.
\end{defn}

\begin{prop}
	Suppose there are $T$-invariant subspaces $V_i$ with bases $B_i$ such that $V = \bigoplus V_i$. Let $A_i = [T_{V_i}]_{B_i}$ and $B = \bigcup B_i$.
	Then
	$$[T]_B = \begin{pmatrix}A_1 & & \\ & \ddots & \\ & & A_k \end{pmatrix}$$
\end{prop}

\begin{not*}
	From now on, block diagonal matrices will be written using $\oplus$ or $\bigoplus$ whenever possible.
	For instance,
	$$A \oplus B := \begin{pmatrix} A & 0 \\ 0 & B \end{pmatrix}$$
\end{not*}

\section{Quotient spaces}

\begin{defn*}
	For a subspace $W$ of $V$, the \textbf{quotient space} $V / W$ consists of the cosets
	$$W + v := \{w + v \mid w \in W\}$$
	where $v \in V$.
	We define addition as
	$$(W + v) + (W + v') := W + (v + v')$$
	and scalar multiplication as
	$$\lambda(W + v) := W + \lambda v$$
	with $v, v' \in V$ and $\lambda \in F$ arbitrary.
\end{defn*}

\begin{prop}
	$V / W$ is a vector space over $F$.
\end{prop}

\begin{prop}
	If $V$ is finite-dimensional, $\dim{V / W} = \dim{V} - \dim{W}$.
\end{prop}

\begin{prop*}
	If $B_W = \{w_1, \ldots, w_r\}$ is a basis of $V$ and $B = B_W \cup \{v_1, \ldots, v_s\}$ is a basis of $V$ then $\bar{B} := \{W + v_1, \ldots, W + v_s\}$ is a basis of $V / W$.
\end{prop*}

\begin{defn*}
	If $W$ is $T$-invariant, the \textbf{quotient map} $\bar{T} : V / W \to V / W$ is
	$$\bar{T}(W + v) := W + T(v)$$
\end{defn*}

\begin{prop}
	With $B$, $B_W$ and $\bar{B}$ defined as above,
	$$[T]_B = \begin{pmatrix} [T_W]_{B_W} & Z \\ 0 & [\bar{T}]_{\bar{B}} \end{pmatrix}$$
	for some $r \times s$ matrix $Z$.
\end{prop}

\begin{cor}
	$c_T(x) = c_{T_W}(x)c_{\bar{T}}(x)$.
\end{cor}

\section{Triangularisation}

\begin{prop}
	The values along the main diagonal of an upper triangular matrix are its eigenvalues.
	The product of two upper triangular matrices is also upper triangular, and its diagonal entries are the products of the corresponding diagonal entries of the factors.
\end{prop}

\begin{thm}
	(Triangularisation Theorem)
	If $c_T(x)$ factorises as a product of linear factors then there is a basis $B$ of $V$ such that $[T]_B$ is upper triangular.
\end{thm}

\noindent To triangularise $T$, choose an eigenvector $w_1$ of $T$ and let $W_1 = \spn{w_1}$.
Let $W + w_2$ be an eigenvector for the quotient map with respect to $W$ and let $W_2 = \spn{\{w_1, w_2\}}$.
Let $W + w_3$ be an eigenvector for the quotient map with respect to $W_2$ and let $W_3 = \spn{\{w_1, w_2, w_3\}}$.
Continue in this fashion until you have $B = \{w_1, \ldots, w_n\}$.
Then $[T]_B$ is upper triangular.

\section{Cayley-Hamilton Theorem}

\begin{thm}
	(Cayley-Hamilton Theorem)
	$c_T(T) = 0$.
\end{thm}

\begin{cor}
	$c_A(A) = 0$.
\end{cor}

\begin{defn*}
	The \textbf{companion matrix} of $p(x) = \sum_{i = 0}^na_ix^i$ where $a_n = 1$ is
	$$C(p) = \begin{pmatrix}
		0 & 0 & 0 & \ldots & 0 & -a_0 \\
		1 & 0 & 0 & \ldots & 0 & -a_1 \\
		0 & 1 & 0 & \ldots & 0 & -a_2 \\
		\vdots & \vdots & \vdots & \ddots & \vdots & \vdots \\
		0 & 0 & 0 & \ldots & 1 & -a_{n - 1}
	\end{pmatrix}$$
\end{defn*}

\section{Polynomials}

\begin{defn*}
	A \textbf{polynomial} over $F$ is an expression of the form
	$$p(x) := \sum_{i = 1}^na_ix^i$$
	with all $a_i \in F$.
	The set of polynomials over $F$ is $F[x]$.
\end{defn*}

\begin{defn*}
	The \textbf{zero polynomial} is $p(x) = 0$.
\end{defn*}

\begin{defn*}
	The degree of a non-zero polynomial is the highest $\deg{p}$ such that $x^{\deg{p}}$ occurs in $p(x)$ with a non-zero coefficient.
	$\deg{0}$ is undefined.
\end{defn*}

\begin{defn*}
	$p(x)$ \textbf{divides} $q(x) \iff \exists r(x) \in F[x] : q(x) = p(x)r(x)$.
	In this case we write $p(x) \mid q(x)$.
\end{defn*}

\begin{not*}
	From now on I will usually write $p$ rather than $p(x)$.
\end{not*}

\begin{prop}
	(Euclidean Algorithm)
	For $f, g \in F[x]$ non-constant,
	$$\exists q, r \in F[x] : f = qg + r$$
	with $r = 0$ or $\deg{r} < \deg{g}$.
\end{prop}

\begin{defn}
	$d \in F[x]$ is a \textbf{greatest common divisor} of $f, g \in F[x]$ both non-zero iff
	\begin{itemize}
		\item $d \mid f$
		\item $d \mid g$
		\item For all $e \in F[x]$, $e \mid f$ and $e \mid g \implies e \mid d$.
	\end{itemize}
	In this case we write $\gcd{(f, g)} = d$.
\end{defn}

\begin{prop}
	For $f, g \in F[x]$ non-zero, $\gcd{(f, g)}$ exists and is unique up to scalar multiplication.
\end{prop}

\begin{defn*}
	$f, g \in F[x]$ are \textbf{co-prime} $\iff \gcd{(f, g)} = 1$.
\end{defn*}

\begin{prop}
	For $f, g \in F[x]$ non-zero, $\exists r, s \in F[x] : \gcd{(f, g)} = rf + sg$.
\end{prop}

\begin{defn*}
	An \textbf{irreducible} polynomial is one which is non-constant and cannot be factorised as a product of polynomials of smaller degree.
\end{defn*}

\begin{prop}
	For $p(x) \in \mathbb{Q}[x]$ monic with integer coefficients, all roots of $p$ in $\mathbb{Q}$ are integers.
	If $p(x)$ is reducible, then $p = ab$ where $a$ and $b$ are also monic with integer coefficients (this statement is called Gauss's Lemma).
\end{prop}

\begin{prop}
	If $p$ is irreducible and $p \mid ab$, either $p \mid a$ or $p \mid b$.
\end{prop}

\begin{cor}
	If $p$ is irreducible and $p \mid \prod g_i$, then $\exists i : p \mid g_i$.
\end{cor}

\begin{thm}
	(Unique Factorization Theorem)
	Any non-constant polynomial can be written as a product of irreducible factors.
	This factorisation is unique up to scalar multiplication of the factors.
\end{thm}

\section{The minimal polynomial of a linear map}

\begin{defn*}
	$m_T(x) \in F[x]$ is a \textbf{minimal polynomial} for $T$ if $m_T(T) = 0$, $m_T$ is monic and there is no polynomial of smaller degree for which the other two conditions hold.
\end{defn*}

\begin{prop}
	$m_T$ is unique.
	For all $p \in F[x]$, $p(T) = 0 \iff m_T \mid p$.
\end{prop}

\begin{prop}
	$m_T \mid c_T$.
	For all $\lambda \in F$, $c_T(\lambda) = 0 \implies m_T(\lambda) = 0$.
\end{prop}

\begin{thm}
	All irreducible factors of $c_T$ divide $m_T$.
\end{thm}

\begin{prop}
	For $T_W$ and $\bar{T}$ as in sections 4 and 5, $m_{T_W} \mid m_T$ and $m_{\bar{T}} \mid m_T$.
\end{prop}

\section{Primary Decomposition}

\begin{thm}
	(Primary Decomposition Theorem)
	Suppose $m_T(x) = \prod_{i = 1}^kf_i(x)^{n_i}$ with the $f_i$ distinct and irreducible.
	Let $V_i := \ker{f_i(T)^{n_i}}$.
	Then $V = \bigoplus_{i = 1}^kV_i$, each $V_i$ is $T$-invariant, and $m_{T_{V_i}} = f_i(x)^{n_i}$.
\end{thm}

\begin{cor}
	$T$ is diagonalisable $\iff m_T$ is a product of distinct linear factors.
\end{cor}

\begin{prop}
	Suppose $g_1, g_2 \in F[x]$ are coprime and $g_1(T)g_2(T) = 0$.
	Let $V_i = \ker{g_i(T)}$.
	Then $V = V_1 \oplus V_2$ and the $V_i$ are $T$-invariant.
	If additionally $m_T = g_1g_2$, then $m_{T_{V_i}} = g_i$.
\end{prop}

\section{Jordan Canonical Form}

\subsection{Definition and properties}

\begin{defn*}
	A \textbf{Jordan block} is an $n \times n$ matrix of the form
	$$J_n(\lambda) := \begin{pmatrix}
		\lambda & 1 & 0 & \ldots & 0 \\
		0 & \lambda & 1 & \ldots & 0 \\
		0 & 0 & \lambda & \ldots & 0 \\
		\vdots & \vdots & \vdots & \ddots & \vdots \\
		0 & 0 & 0 & \ldots & \lambda
	\end{pmatrix}$$
\end{defn*}

\begin{prop}
	Let $J = J_n(\lambda)$.
	\begin{itemize}
		\item $c_J(x) = m_J(x) = (x - \lambda)^n$
		\item $\lambda$ is the only eigenvalue of $J$
		\item $a(\lambda) = n$
		\item $g(\lambda) = 1$
		\item $J - \lambda I = J_n(0)$
		\item $(J - \lambda I)e_1 = 0$ and $(J - \lambda I)e_k = e_{k - 1}$
		\item $(J - \lambda I)^n = 0$
		\item $\rank{(J - \lambda I)^i} = n - i$
		\item $(J - \lambda I)^ie_k = e_{k - i}$
	\end{itemize}
\end{prop}

\begin{prop}
	Let $A = \bigoplus A_i$.
	\begin{itemize}
		\item $c_A = \prod c_{A_i}$
		\item $m_A = \lcm{\{m_{A_i}\}}$
		\item For any eigenvalue $\lambda$ of $A$, $\dim{E_{\lambda}(A)} = \sum \dim{E_{\lambda}(A_i)}$
		\item For any $q(x) \in F[x]$, $q(A) = \bigoplus q(A_i)$
	\end{itemize}
\end{prop}

\begin{thm}
	If $c_A$ is a product of linear factors, then $A$ is similar to a matrix of the form
	$$J = \bigoplus_{i = 1}^kJ_{n_i}(\lambda_i)$$
	where the $\lambda_i$ are the eigenvalues of $A$, not necessarily distinct.
	$J$ is unique up to reordering the Jordan blocks.
\end{thm}

\begin{defn*}
	The \textbf{Jordan Canonical Form} of $A$ is $J$ as defined above.
\end{defn*}

\begin{prop}
	Suppose $J$ has $a$ blocks associated with an eigenvalue $\lambda$, with sizes $n_1, \ldots, n_a$.
	Then:
	\begin{itemize}
		\item $\sum_{i = 1}^an_i = a(\lambda)$
		\item $a = g(\lambda)$
		\item $\max{\{n_i\}} = \max{\{r : (x - \lambda)^r \mid m_A(x)\}}$ (the size of the largest block is the exponent of $(x - \lambda)$ as a factor of $m_A$)
	\end{itemize}
\end{prop}

\subsection{Steps in the proof of Theorem 11.3 that are numbered as though they were important even though they are just special cases of Theorem 11.3}

\begin{thm}
	The JCF is unique up to reordering the blocks.
\end{thm}

\begin{thm}
	If $c_T$ is a product of linear factors, then there exists a basis $B$ of $V$ such that $[T]_B$ is in JCF.
\end{thm}

\begin{defn*}
	$T$ is \textbf{nilpotent} $\iff \exists k : T^k = 0$.
\end{defn*}

\begin{thm}
	If $T$ is nilpotent, then there exists a basis $B$ of $V$ such that $[S]_B = \bigoplus J_{n_i}(0)$
\end{thm}

\begin{cor}
	In the situation above, $[T + \lambda I_V]_B = \bigoplus J_{n_i}(\lambda)$.
\end{cor}

%\subsection{How to compute the JCF}
%
%\noindent First split up $V$ into $V_i$ as in the Primary Decomposition Theorem, so that each $V_i$ corresponds to a distinct eigenvalue $\lambda_i$.
%Let $S_i := T_{V_i} - \lambda_iI$, which will be nilpotent.

\section{Cyclic Decomposition and Rational Canonical Form}

\subsection{Cyclic Decomposition}

\begin{defn*}
	Let $V$ be finite-dimensional.
	The \textbf{$T$-cyclic subspace} of $V$ generated by $v \in V$ is
	\begin{align*}
		Z(v, T) &:= \{f(T)(v) : f(x) \in F[x]\}\\ 
				&= \spn\{v, T(v), T^2(v), \ldots\}
	\end{align*}
	Note that $Z(v, T)$ is $T$-invariant.
	We abbreviate the restriction $T_{Z(v, T)}$ to $T_v$.
\end{defn*}

\begin{defn*}
	The \textbf{$T$-annihilator} of $v$ and $Z(v, T)$ is $m_v$, the monic polynomial of smallest degree such that $m_v(T)(v) = 0$.
	It can be constructed as follows:
	Let $k$ be as small as possible such that $\{v, T(v), T^2(v), \ldots, T^k(v)\}$ is not linearly independent.
	Then $T^k(v) = -a_0v - a_1T(v) - \ldots - a_{k - 1}T^{k - 1}(v)$, and $m_v$ is the polynomial with coefficients $a_i$.
\end{defn*}

\begin{prop}
	With $k$ defined as above, $B = \{v, T(v), T^2(v), \ldots, T^{k - 1}(v)\}$ is a basis of $Z(v, T)$. $[T_v]_B = C(m_v)$. $m_{T_v} = m_v$.
\end{prop}

\begin{thm}
	(Cyclic Decomposition Theorem)
	Let $V$ be finite-dimensional.
	If $m_T(x) = f(x)^k$ with $f$ irreducible, then there exist unique numbers $r$ and $k_1, \ldots, k_r$ such that
	\begin{align*}
		k &= k_1 \geq k_2 \geq \ldots \geq k_r\\
		V &= \bigoplus_{i = 1}^rZ(v_i, T)\\
		m_{v_i} &= f(x)^{k_i}\quad i = 1, \ldots, r
	\end{align*}
	for some $v_1, \ldots, v_r \in V$
\end{thm}

\begin{cor}
	If $T$ is as above, there exists a basis $B$ of $V$ such that
	$$[T]_B = \bigoplus_{i = 1}^rC(f(x)^{k_i})$$
	with $k = k_1 \geq k_2 \geq \ldots \geq k_r$ uniquely determined by $T$.
\end{cor}

\begin{cor}
	If $m_A(x) = x^k$, then
	$$A \sim \bigoplus_{i = 1}^rC(x^{k_i})$$
	with $k = k_1 \geq k_2 \geq \ldots \geq k_r$ uniquely determined by $A$.
\end{cor}

\subsection{RCF}

\begin{thm}
	(Rational Canonical Form)
	Let $V$ be finite-dimensional.
	If
	$$m_T(x) = \prod_{i = 1}^tf_i(x)^{k_i}$$
	with the $f_i$ distinct and irreducible, then there are unique numbers
	\begin{align*}
		   r_1, &\ldots, r_t\\
		k_{11}, &\ldots, k_{1r_1}\\
		k_{22}, &\ldots, k_{2r_2}\\
				&\vdots\\
		k_{t1}, &\ldots, k_{tr_t}
	\end{align*}
	such that
	\begin{align*}
		k_i &= k_{i1} \geq k_{i2} \geq \ldots \geq k_{ir_i}\quad i = 1, \ldots, t\\
		[T]_B &= \bigoplus_{i = 1}^t\left(\bigoplus_{j = 1}^{r_i}C\left(f_i(x)^{k_{ij}}\right)\right)
	\end{align*}
	for some basis $B$ of $V$.
\end{thm}

\begin{cor}
	If the minimal polynomial of $A$ is of the form of $m_T$ above, then $A$ is similar to a unique matrix of the form of $[T]_B$ above.
\end{cor}

\begin{defn*}
	The \textbf{rational canonical form} of $A$ is the matrix described in the corollary above.
\end{defn*}

\begin{prop}
	This proposition is not in the official notes, but it should be.
	Let $f_i$ be one of the factors of $m_A$.
	Consider the blocks of the RCF of $A$ associated with $f_i$, which are
	$$C(f_i(x)^{k_{i1}}) \oplus \ldots \oplus C(f_i(x)^{k_{ir_i}})$$
	with the exponents defined as in Theorem 12.5.
	Then:
	\begin{itemize}
		\item $\sum_{j=1}^{r_i}k_{ij} = \max{\{k : f^k \mid c_A\}}$
		\item $r_i = \frac{n - \rank{(f(A))}}{\deg{f}}$
		\item $k_{i1} = k_i = \max{\{k : f^k \mid m_A\}}$
	\end{itemize}
	The first and third statements are trivial; I included them to emphasise the parallels to Proposition 11.4.
	The second statement was proved in a live lecture. (TODO: add link)
\end{prop}

%\subsection{How to compute the RCF}
%
%\noindent I wish I knew.

\section{The dual space}

\begin{defn*}
	A \textbf{linear functional} on $V$ is a linear map $\phi : V \to F$.
\end{defn*}

\begin{defn*}
	The \textbf{dual space} of $V$ is $V^{\ast}$, the set of linear functionals on $V$.
	Addition and scalar multiplication are defined as usual using the equivalent operations on $V$.
	This is a vector space over $F$.
\end{defn*}

\begin{defn*}
	The \textbf{Kronecker delta} is
	$$\delta_{ij} := \begin{cases} 1 & i = j \\ 0 & i \neq j \end{cases}$$
\end{defn*}

\begin{prop}
	Let $n = \dim{V}$ and $B = \{v_1, \ldots, v_n\}$ be a basis of $V$.
	Define
	$$\phi_i(v_j) = \delta_{ij}\quad i = 1, \ldots, n, j = 1, \ldots, n$$
	meaning that for any $v = \sum\alpha_jv_j$
	$$\phi_i(v) = \alpha_i$$
	Then $\{\phi_1, \ldots, \phi_n\}$ is a basis of $V^{\ast}$, and therefore $\dim{V^{\ast}} = \dim{V}$.
\end{prop}

\begin{defn*}
	Let $V$ be finite-dimensional.
	The \textbf{annihilator} of $X \subseteq V$ is
	$$X^0 := \{\phi \in V^{\ast} : \phi(x) = 0\quad \forall x \in X\}$$
	This is a subspace of $V^{\ast}$.
\end{defn*}

\begin{prop}
	$\dim{W^0} = \dim{V} - \dim{W}$.
\end{prop}

\section{Inner Product Spaces}

\subsection{Definition and matrix representation}

\begin{not*}
	From now on let $F$ be either $\mathbb{R}$ or $\mathbb{C}$, such that there is a conjugate map $\bar{\cdot} : F \to F$.
\end{not*}

\begin{defn*}
	An \textbf{inner product} on $V$ is a map $(\cdot, \cdot) : V \times V \to F$ where
	\begin{itemize}
		\item $(\lambda u + \mu v, w) = \lambda(u, w) + \mu(v, w)$
		\item $\overline{(v, w)} = (w, v)$
		\item $v \neq 0 \implies (v, v) > 0$
	\end{itemize}
	$V$ together with $(\cdot, \cdot)$ is an \textbf{inner product space}.
\end{defn*}

\begin{defn*}
	$A$ is \textbf{Hermitian} $\iff A^T = \bar{A}$.
\end{defn*}

\begin{defn*}
	For $\dim{V} = n$, the \textbf{matrix of an inner product} for some basis $B = \{v_1, \ldots, v_n\}$ has entries $a_{ij} = (v_i, v_j)$.
	This matrix $A$ is Hermitian, and for any $v, w \in V$, $(v, w) = [v]_B^TA\bar{[w]}_B$, so $A$ defines $(\cdot, \cdot)$ uniquely.
\end{defn*}

\begin{defn*}
	A Hermitian matrix $A$ is \textbf{positive-definite} iff
	$$\forall x \in F^n \setminus \{0\},\  x^TA\bar{x} > 0$$
\end{defn*}

\subsection{Geometric functions}

\begin{defn*}
	The \textbf{length} of $u \in V$ is $\length{u} := \sqrt{(u, u)}$.
	The \textbf{distance} between $u, v\in V$ is $d(u, v) := \length{u - v}$.
	$u$ is a \textbf{unit vector} $\iff \length{u} = 1$.
\end{defn*}

\begin{prop}
	For any $u, v, w$ in an inner product space:
	\begin{itemize}
		\item $\lvert (u, v) \rvert \leq \length{u} \length{v}$
		\item $\length{u + v} \leq \length{u} + \length{v}$
		\item $\length{u - v} \leq \length{u - w} + \length{w - v}$
	\end{itemize}
\end{prop}

\subsection{Dual space}

\begin{defn*}
	$f_v \in V^{\ast}$ is $f_v(w) = (w, v)$.
\end{defn*}

\begin{defn*}
	$\bar{V}$ is a vector space over $F$ with the same vectors and addition as $V$, but scalar addition defined as $\lambda \ast v := \bar{\lambda}v$, where the operation on the right hand side is scalar multiplication from $V$.
\end{defn*}

\begin{prop}
	 $\pi(v) := f_v$ is a vector space isomorphism $\pi : \bar{V} \to V^{\ast}$.
\end{prop}

\begin{cor}
	$\forall f \in V^{\ast},\ \exists! v \in V : f = f_v$.
\end{cor}

\subsection{Orthogonality}

\begin{defn*}
	$u$ and $v$ are \textbf{orthogonal} $\iff (u, v) = 0$.
	A set of vectors is \textbf{orthogonal} iff any pair of distinct vectors is orthogonal.
	An orthogonal set of unit vectors is \textbf{orthonormal}.
\end{defn*}

\begin{defn*}
	For $X \subseteq V$
	$$X^{\bot} := \{u \in V : (u, w) = 0\quad \forall w \in W\}$$
	This is a subspace of $V$.
\end{defn*}

\begin{prop}
	If $V$ is finite-dimensional and $W$ is a subspace of $V$, then $V = W \oplus W^{\bot}$.
\end{prop}

\begin{thm}
	Any finite-dimensional inner product space has an orthonormal basis, and any orthonormal set can be extended to an orthonormal basis.
\end{thm}

\begin{prop*}
	(Gram-Schmidt Process)
	Given a basis $\{v_1, \ldots, v_n\}$, define
	\begin{align*}
		u_1 &:= \frac{v_1}{\length{v_1}}\\
		w_i &:= v_i - \sum_{j = 1}^{i - 1}(v_i, u_j)u_j\\
		u_i &:= \frac{w_i}{\length{w_i}}
	\end{align*}
	Then $\{u_1, \ldots, u_n\}$ is an orthonormal basis.
\end{prop*}

\begin{prop}
	For $v \in V$ and an orthonormal basis $\{u_1, \ldots, u_n\}$ for $V$, $v = \sum_{i = 1}^n(v, u_i)u_i$ and $\length{v}^2 = \sum_{i = 1}^n\lvert (v, u_i) \rvert^2$.
\end{prop}

\begin{defn*}
	The \textbf{projection} of $v \neq 0$ along $w \neq 0$ is $\frac{(v, w)}{(w, w)}v$
\end{defn*}

\begin{defn*}
	The \textbf{orthogonal projection map} along a subspace $W$ is $\pi_W(w + w') = w$ where $w \in W$ and $w' \in W^{\bot}$ (recall that for any $v$ there are unique $w \in W, w' \in W^{\bot}$ such that $v = w + w'$, so this is well-defined).
\end{defn*}

\begin{prop}
	$\length{v - \pi_W(v)} = \min{\{\length{v - w} \mid w \in W\}}$ i.e.\@ $\pi_W(v)$ is the closest vector to $v$ in $W$. If $\{u_1, \ldots, u_r\}$ is an orthonormal basis of $W$, then $\pi_W(v) = \sum_{i = 1}^r(v, v_i)v_i$.
\end{prop}

\begin{prop}
	Let $E = \{e_1, \ldots, e_n\}$ and $F = \{f_1, \ldots, f_n\}$ be two orthonormal bases of $V$.
	Let $P = (p_{ij})$ be the change of basis matrix, i.e.\@ the matrix such that $f_i = \sum_{j = 1}^np_{ji}e_j$.
	Then $P^T\bar{P} = I$.
\end{prop}

\begin{defn*}
	$P$ is \textbf{orthogonal} $\iff P$ is real and $P^TP = I$.
	${P}$ is \textbf{unitary} $\iff P$ is complex and $P^T\bar{P} = I$.
	Each of these types is a group, called a \textbf{classical group}.
\end{defn*}

\section{Linear maps on inner product spaces}

\subsection{Definition and adjoints}

\begin{prop}
	Let $V$ be finite-dimensional
	For any linear map $T : V \to V$ there is a unique linear map $T^{\ast} : V \to V$ such that
	$$\forall u, v \in V,\ (T(u), v) = (u, T^{\ast}(v))$$
\end{prop}

\begin{defn*}
	$T^{\ast}$ is the \textbf{adjoint} of $T$. $T$ is \textbf{self-adjoint} $\iff T = T^{\ast}$.
\end{defn*}

\begin{prop}
	If $B$ is an orthonormal basis then $[T^{\ast}]_B = \bar{[T]}_B^T$ 
\end{prop}

\begin{thm}
	(Spectral Theorem)
	If $T$ is self-adjoint then $V$ has an orthonormal basis of $T$-eigenvectors.
\end{thm}

\begin{cor}
	If $A$ is real and symmetric, there exists an orthogonal $P$ such that $P^{-1}AP$ is diagonal.
	If $A$ is complex and Hermitian, there exists a unitary $P$ such that $P^{-1}AP$ is diagonal.
\end{cor}

\begin{lem}
	If $T$ is self-adjoint, all its eigenvalues are real, eigenvectors to distinct eigenvalues are orthogonal, and if $W$ is $T$-invariant then so is $W^{\bot}$.
\end{lem}

\subsection{How to find an orthonormal basis}

If $T$ is self-adjoint, we can find the basis described in Theorem 15.3 as follows:
For each eigenvalue of $T$, use the Gram-Schmidt Process to find an orthonormal basis of the eigenspace.
Then take the union of all these bases.

\section{Bilinear and Quadratic Forms}

\subsection{Bilinear forms}

\begin{not*}
	$F$ is now back to being arbitrary.
\end{not*}

\begin{defn*}
	A \textbf{bilinear form} on $V$ is $(\cdot, \cdot) : V \times V \to F$ where
	\begin{align*}
		(\alpha v_1 + \beta v_2, w) &= \alpha(v_1, w) + \beta(v_2, w)\\
		(v, \alpha w_1 + \beta w_2) &= \alpha(v, w_1) + \beta(v, w_2)
	\end{align*}
\end{defn*}

\begin{defn*}
	For $\dim{V} = n$, the \textbf{matrix of a bilinear form} for some basis $B = \{v_1, \ldots, v_n\}$ has entries $a_{ij} = (v_i, v_j)$.
	Then for any $v, w \in V$, $(v, w) = [v]_B^TA[w]_B$, so $A$ defines $(\cdot, \cdot)$ uniquely.
\end{defn*}

\subsection{Symmetry and skew-symmetry}

\begin{defn*}
	$(\cdot, \cdot)$ is \textbf{symmetric} $\iff \forall v, w \in V,\ (v, w) = (w, v)$.
	$(\cdot, \cdot)$ is \textbf{skew-symmetric} $\iff \forall v, w \in V,\ (v, w) = -(w, v)$.
\end{defn*}

\begin{defn*}
	The \textbf{characteristic} of $F$ is the smallest $\chr{F}$ such that $\chr{F} = 0$ in $F$, or 0 if there is no such number.
	$\chr{\mathbb{R}} = \chr{\mathbb{C}} = 0$.
	$\chr{\mathbb{F}_p} = p$.
\end{defn*}

\begin{lem}
	$\chr{F} \neq 2 \implies \forall v \in V,\ (v, v) = 0$
\end{lem}

\begin{thm}
	$(\cdot, \cdot)$ is symmetric or skew-symmetric $\iff \forall v, w \in V,\ ((v, w) = 0 \iff (w, v) = 0)$.
\end{thm}

\begin{defn*}
	For $X \subseteq V$
	$$X^{\bot} := \{u \in V : (u, w) = 0\quad \forall w \in W\}$$
	This is a subspace of $V$.
\end{defn*}

\begin{defn*}
	$(\cdot, \cdot)$ is \textbf{non-degenerate} $\iff V^{\bot} = \{0\}$.
\end{defn*}

\begin{prop}
	Let $V$ be finite-dimensional and $(\cdot, \cdot)$ be non-degenerate and symmetric or skew-symmetric.
	Define $f_v(u) := (v, u)$.
	Then $\phi(v) := f_v$ is an isomorphism $\phi : V \to V^{\ast}$.
	Furthermore, if $W$ is a subspace of $V$, then $\dim{W^{\bot}} = \dim{V} - \dim{W}$.
\end{prop}

\begin{defn*}
	$A$ and $B$ both $n \times n$ over $F$ are \textbf{congruent} $\iff$ there exists $P$ invertible over $F$ such that $B = P^TA^P$.
	Two bilinear forms are \textbf{equivalent} $\iff$ their matrices are congruent.
\end{defn*}

\begin{thm}
	Let $V$ be finite-dimensional, $\chr{F} \neq 2$, and $(\cdot, \cdot)$ non-degenerate and skew-symmetric.
	Then $\dim{V}$ is even, and there is a basis $B = {e_1, f_1, \ldots, e_m, f_m}$ for $2m = n$ such that the matrix of $(\cdot, \cdot)$ with respect to $B$ is
	$$\bigoplus_{i = 1}^m\begin{pmatrix} 0 & 1 \\ -1 & 0 \end{pmatrix}$$
	Equivalently,
	$$(e_i, f_i) = -(f_i, e_i) = 1$$
	and if $i \neq j$
	$$(e_i, e_j) = (f_i, f_j) = (e_i, f_j) = (f_j, e_i) = 0$$
\end{thm}

\begin{cor}
	If $A$ is invertible and skew-symmetric and $\chr{F} \neq 2$, then $A$ is congruent to a matrix of the form above.
\end{cor}

\begin{thm}
	Let $V$ be finite-dimensional, $\chr{F} \neq 2$, and $(\cdot,\cdot)$ non-degenerate and symmetric.
	Then there is an orthogonal basis $B = \{u_1, \ldots, u_n\}$ such that $(u_i, u_j) = 0$ when $i \neq j$ and $(u_i, u_i) \neq 0$.
	The matrix of $(\cdot, \cdot)$ with respect to this $B$ is $\diag{(u_1, u_1), \ldots, (u_n, u_n)}$.
\end{thm}

\begin{cor}
	If $A$ is invertible and symmetric and $\chr{F} \neq 2$, then $A$ is congruent to a diagonal matrix.
\end{cor}

\subsection{Quadratic forms}

\begin{defn*}
	A \textbf{quadratic form} on $V$ is $Q : V \to F$ with $Q(v) := (v, v)$, where $(\cdot, \cdot)$ is symmetric.
	$Q$ is \textbf{non-degenerate} $\iff (\cdot, \cdot)$ is non-degenerate.
\end{defn*}

\begin{defn*}
	$Q, Q'$ are \textbf{equivalent} $\iff$ there is an invertible matrix $P$ such that $Q(Py) = Q'(y)$ $\iff$ their matrices are congruent.
	Then letting $x = Py$, we have $Q(x) = (Py)^TA(Py) = y^TP^TAPy = Q'(y)$ where $A$ is the matrix of $Q$.
\end{defn*}

\begin{thm}
	Let $V = F^n$ and $Q$ be non-degenerate.
	If $F = \mathbb{C}$ then $Q$ is equivalent to $Q_0(x) := \sum_{i = 1}^nx_i^2$, whose matrix is $I_n$.
	If $F = \mathbb{R}$ then $Q$ is equivalent to a unique $Q_{p,q} := \sum_{i = 1}^px_i^2 - \sum_{i = p + 1}^{p + q}x_i^2$ where $p + q = n$, whose matrix is $I_p \oplus -I_q$.
	If $F = \mathbb{Q}$ then there are infinitely many inequivalent non-degenerate quadratic forms.
\end{thm}

\subsection{Applications}

\begin{defn*}
	Let $(\cdot, \cdot)$ be non-degenerate and symmetric or skew-symmetric.
	$T$ is an \textbf{isometry} of $(\cdot, \cdot) \iff \forall u, v \in V,\ (T(u), T(v)) = (u, v)$.
	The set of isometries $I(V, (\cdot, \cdot))$ is a subgroup of $GL(V)$.
\end{defn*}

\end{document}
