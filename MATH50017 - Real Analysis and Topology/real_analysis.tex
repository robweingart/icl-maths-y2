\documentclass[12pt]{article}
\usepackage[a4paper, margin=1in]{geometry}
\usepackage{amsmath,amssymb,amsfonts,amsthm,mathtools}

\newtheorem{thm}{Theorem}[section]
\newtheorem{lem}[thm]{Lemma}
\newtheorem{cor}[thm]{Corollary}
\newtheorem{prop}[thm]{Proposition}
\newtheorem*{prop*}{Proposition}

\theoremstyle{definition}
\newtheorem{defn}{Definition}[section]
\newtheorem*{defn*}{Definition}
\newtheorem*{not*}{Notation}

\DeclarePairedDelimiter\abs{\lvert}{\rvert}
\DeclarePairedDelimiter\length{\lVert}{\rVert}

\renewcommand{\d}{{\textnormal{d}}}
\newcommand{\R}{\mathbb{R}}
\newcommand{\N}{\mathbb{N}}

\title{MATH50001 Analysis 2 Term 1\\{\Large also known as}\\MATH50017 Real Analysis and Topology for JMC}
\author{Notes by Robert Weingart}
\date{}

\begin{document}

\maketitle

\tableofcontents

\section*{General notes}

This document is a concise summary of the official course notes.
It includes every theorem, lemma, proposition, corollary and definition, without any proofs or exercises.
The numbering should match the official notes, but notation has been changed in many places to be more consistent.\\

\noindent Notation which is used throughout an entire section is defined only once at the start rather than in every statement.
When a symbol has the same meaning in multiple consecutive statements, the definition is not repeated as it should be clear from the context.\\

\noindent Unnumbered definitions correspond to bolded terms that are not given numbers.
Unnumbered propositions are results from exercises that I though were important enough to include.

\section{Differentiation in higher dimensions}

\subsection{Euclidean spaces}

\subsubsection{Preliminaries from analysis 1}

\begin{defn*}
	Some definitions from year 1:
	\begin{align*}
		\mathbb{N} &:= \{1, 2, 3, \ldots\}\\
		\mathbb{Z} &:= \{\ldots, \}
	\end{align*}
\end{defn*}

\subsubsection{Euclidean space of dimension $n$}

\begin{defn*}
	The \textbf{$n$-dimensional Euclidean space} $\mathbb{R}^n$ is the set of all ordered $n$-tuples of members (\textbf{coordinates}) of $\mathbb{R}$.
	In this module, the $i$th coordinate of $x \in \mathbb{R}^n$ is denoted $x^i$.
	$\mathbb{R}^n$ is a vector space over $\mathbb{R}$, with addition and scalar multiplication applied to each element.
\end{defn*}

\begin{defn*}
	The \textbf{inner product} on $\mathbb{R}^n$ is
	\begin{align*}
		\langle\cdot,\cdot\rangle : \mathbb{R}^n \times \mathbb{R}^n \to \mathbb{R}\\
		\langle x, y\rangle := \sum_{i = 1}^nx^iy^i
	\end{align*}
\end{defn*}

\begin{defn*}
  The \textbf{length} or \textbf{norm} on $\R^n$ is
  \begin{align*}
    \length{\cdot} : \R^n \to [0, \infty)\\
    \length{x} := \sqrt{\langle x, x \rangle}
  \end{align*}
\end{defn*}

\begin{prop*}
  For all $x, y \in \R^n$, $\lambda \in \R$
  \begin{itemize}
    \item $\length{x} \geq 0$
    \item $\length{x} = 0 \iff x = 0$
    \item $\length{\lambda x} = \abs{\lambda}\length{x}$
    \item Triangle inequality: $\length{x + y} \leq \length{x} + \length{y}$
  \end{itemize}
\end{prop*}

\subsubsection{Convergence of sequences in Euclidean spaces}

\begin{defn}
  A sequence $(x_i) \subset \R^n$ \textbf{converges to} $x \in \R $ iff
  $$\forall \epsilon > 0,\ \exists N \in \N : \forall i \geq N,\ \length{x_i - x} < \epsilon$$
\end{defn}

\begin{prop}
  $x_i \to x \iff \forall j \in \{1, \ldots, n\},\ x_i^j \to x^j$ (note the use of superscript to mean component extraction, not exponentiation)
\end{prop}

\subsection{Continuity}

\subsubsection{Open sets in Euclidean spaces}

\begin{defn*}
  The \textbf{open ball} of radius $r \in \R$ about $x \in \R^n$ is
  $$B_r(x) := \{ y \in \R : \length{y - x} < r \}$$
\end{defn*}

\begin{defn}
  $U \subseteq \R^n$ is \textbf{open in} $\R^n \iff \forall x \in U,\ \exists r > 0 : B_r(x) \subseteq U$.
\end{defn}

\subsubsection{Continuity at a point, and continuity on an open set}

\begin{defn}
  For $A$ open, $f : A \to \R$ is \textbf{continuous at} $p \in A$ iff
  $$\forall \epsilon > 0,\ \exists \delta > 0 : \forall x \in A,\ \length{x - p} < \delta \implies \length{f(x) - f(p)} < \epsilon$$
  $f$ is \textbf{continuous on} $A \iff f$ is continuous at all $p \in A$.
\end{defn}

\begin{thm}
  For $A \subset \R^n$ and $B \subset \R^m$ both open, if $f : A \to B$ is continuous at $p$ and $g : B \to \R^l$ is continuous at $f(p)$ then $g \circ f$ is continuous at $p$.
\end{thm}

\begin{defn}
  For $A$ open and $f : A \to \R^m$, the \textbf{limit of $f$ as $x$ tends to $p \in A$} is $q \in \R^m$ where
  $$\forall \epsilon > 0,\ \exists \delta > 0 : \forall x \in A,\ 0 < \length{x - p} < \delta \implies \length{f(x) - q} < \epsilon$$
  which we write $\lim_{x \to p}f(x) = q$.
\end{defn}

\begin{prop*}
  $f$ is continuous at $p \iff \lim_{x \to p}f(x) = f(p)$.
\end{prop*}

\begin{thm}
  The limit can be distributed over addition and multiplication, and also over division if the limit of the denominator is not 0.
\end{thm}

\begin{cor}
  If $f : A \to \R$ and $g : A \to R$ are continuous at $p$ then so are $f + g$ and $fg$, and so is $\frac{f}{g}$ assuming $g(p) \neq 0$.
\end{cor}

\subsection{Derivative of a map of Euclidean spaces}

\subsubsection{Derivative as a linear map}

\begin{lem}
  $f : (a, b) \to \R$ is differentiable at $p \in (a, b) \iff$ there exists a map $A_{\lambda}(x) = \lambda(x - p) + f(p)$ with $\lambda \in \R$ such that
  $$\lim_{x \to p}\frac{\abs{f(x) - A_{\lambda}(x)}}{\abs{x - p}} = 0$$
\end{lem}

\begin{defn}
  For $\Omega$ open, $f : \Omega \to \R^m$ is \textbf{differentiable at} $p \in \Omega$ iff there is a linear map $\Lambda : \R^n \to \R^m$ such that
  $$\lim_{x \to p}\frac{\length{f(x) - (\Lambda(x - p) + f(p))}}{\length{x - p}} = 0$$
  Then the \textbf{derivative of $f$ at $p$} or (\textbf{total derivative} or \textbf{differential}) is $Df(p) := \Lambda$.
\end{defn}

\begin{lem}
  $f$ is differentiable at $p \implies f$ is continuous at $p$.
\end{lem}

\begin{thm}
  If the derivative exists, it is unique.
\end{thm}

\subsubsection{Chain rule}

\begin{thm}
  For $A \in \R^n$ and $B \in \R^m$ both open, $g : A \to B$ differentiable at $p$ and $f : B \to \R^l$ differentiable at $g(p)$, $f \circ g$ is differentiable at $p$ with
  $$D(f \circ g)(p) = Df(g(p)) \circ Dg(p)$$
\end{thm}

\subsection{Directional derivatives}

\subsubsection{Rates of change and partial derivatives}

\begin{defn}
  For $f$ differentiable at $p$ and $\length{v} = 1$, the \textbf{directional derivative} of $f$ at $p$ in the direction $v$ is
  $$\frac{\partial f}{\partial v}(p) := Df(p)(v) = \lim_{t \to 0}\frac{f(p + vt) - f(p)}{t}$$
  The \textbf{partial derivatives} of $f$ at $p$ are $D_if(p) := \frac{\partial f}{\partial e_i}(p)$ for $i = 1, \ldots, n$.
\end{defn}

\begin{thm}
  For $f$ differentiable at $p$,
  $$Df(p) = \begin{pmatrix}
    D_1f^1(p) & \ldots & D_nf^1(p) \\
    \vdots    & \ddots & \vdots    \\
    D_1f^m(p) & \ldots & D_nf^m(p)
  \end{pmatrix}$$
\end{thm}

\begin{cor}
  If $g : A \to B$ is differentiable at $p$ and $f : B \to \R^l$ is differentiable at $g(p)$ then
  $$D(f \circ g)(p) = \begin{pmatrix}
    D_1f^1(g(p)) & \ldots & D_mf^1(g(p)) \\
    \vdots    & \ddots & \vdots    \\
    D_1f^l(g(p)) & \ldots & D_mf^l(g(p))
  \end{pmatrix}\begin{pmatrix}
    D_1g^1(p) & \ldots & D_ng^1(p) \\
    \vdots    & \ddots & \vdots    \\
    D_1g^m(p) & \ldots & D_ng^m(p)
  \end{pmatrix}$$
\end{cor}

\begin{lem}
  If $f$ is differentiable in $\Omega$ open and has a local minimum or maximum at $p \in \Omega$ then $Df(p) = 0$.
\end{lem}

\subsubsection{Relation between partial derivatives and differentiability}

\begin{thm}
  If the partial derivatives of $f : \Omega \to \R^m$ all exist in $\Omega$ and are all continuous at $p$ then $f$ is differentiable at $p$.
\end{thm}

\subsection{Higher derivatives}

\subsubsection{Higher derivatives as linear maps}

\begin{defn}
  $f$ is \textbf{continuously differentiable} iff $Df : \Omega \to \R^{mn}$ is continuous, where the $m \times n$ matrix representing $Df(p)$ is converted into an $mn$-dimensional vector.
\end{defn}

\begin{defn}
  The \textbf{second derivative} of $f$ is $DDf : \Omega \to (\R^n \to \R^{m \times n})$ (equivalently, $DDf : \Omega \to \R^n \to \R^n \to \R^m$).
  It is the derivative of $Df : \Omega \to \R^{mn}$.
  For $p \in \Omega$, $DDf(p)$ is a linear map from $n$-dimensional vectors to $m \times n$ matrices.
  This extends to higher derivatives.
\end{defn}

\subsubsection{Symmetry of mixed partial derivatives}

\begin{thm}
  (Schwartz)
  If $f$ is differentiable in $\Omega$ and $D_iD_jf$ and $D_jD_if$ exist and are continuous in $\Omega$ then $D_iD_jf(p) = D_jD_if(p)$ for all $p \in \Omega$.
\end{thm}

\subsubsection{Taylor's theorem}

\begin{not*}
  In this section $\alpha$ is a \textbf{multi-index}, an $n$-tuple of non-negative integers.
  We define:
  \begin{align*}
    \abs{\alpha} &:= \alpha_1 + \alpha_2 + \ldots + \alpha_n\\
    D^{\alpha}f &:= D_1^{\alpha_1}D_2^{\alpha_2}\ldots D_n^{\alpha_n}f\\
    v^{\alpha} &:= (v^1)^{\alpha_1}(v^2)^{\alpha_2}\ldots(v^n)^{\alpha_n}\\
    \alpha! &:= \alpha_1!\alpha_2!\ldots\alpha_n!
  \end{align*}
  Notice that the nested superscripts in the third equation have different meanings; each factor is the $i$th component raised to the power of $\alpha_i$.
\end{not*}

\begin{thm}
  (Taylor)
  For $p \in \R^n$, $h \in B_r(0)$, and $f : B_r(p) \to \R$ which is $k$-times continuously differentiable on $B_r(p)$,
  $$f(p + h) = \sum_{\abs{\alpha} \leq k - 1}\frac{h^{\alpha}}{\alpha!}D^{\alpha}f(p) + R_k(p, h)$$
  where
  $$R_k(p, h) = \sum_{\abs{\alpha} = k}\frac{h^{\alpha}}{\alpha!}D^{\alpha}f(x)$$
  for some $x$ with $0 < \length{x - p} < \length{h}$.
  Each sum is over all $n$-multi-indices $\alpha$ with the specified property.
\end{thm}

\subsection{Inverse and Implicit function theorems}

\subsubsection{Inverse function theorem}

\begin{thm}
  (Inverse Function)
  For $f$ continuously differentiable on $\Omega$ and $q \in \Omega$ such that $Df(q)$ is invertible, there exist $U \subset \Omega$ and $V \subset \R^n$ both open such that $q \in U$, $f(q) \in V$, $f : U \to V$ is bijective, $f^{-1} : V \to U$ is continuously differentiable, and for all $y \in V$,
  $$Df^{-1}(y) = \left(Df(f^{-1}(y))\right)^{-1}$$
\end{thm}

\begin{defn*}
  $f : \Omega \to \Omega'$ is a \textbf{$C^1$-diffeomorphism} iff $f$ is bijective, continuously differentiable, and $Df(x)$ is invertible for all $x \in \Omega$.
\end{defn*}

\subsubsection{Implicit Function Theorem}

\begin{thm}
  (Implicit Function simplified)
  For $\Omega \subset \R^2$ open, $f : \Omega \to \R$ continuously differentiable, and $(x', y') \in \Omega$ with $f(x', y') = 0$ and $D_2f(x', y') \neq 0$, there exist $A, B \subset \R$ both open and $g : A \to B$ continuously differentiable where $x' \in A$, $y' \in B$, and
  $$\forall y \in B,\ (\exists x \in A : f(x, y) = 0) \iff (\exists x \in A : y = g(x))$$
\end{thm}

\subsubsection{* Sketch of the proof of the Implicit Function Theorem}

Not examinable.

\subsubsection{The general from of the Implicit Function Theorem}

\begin{thm}
  For $\Omega \subset \R^n$ and $\Omega' \subset \R^m$ both open, $f : \Omega \times \Omega' \to \R^m$ continuously differentiable everywhere, and $p \in \Omega \times \Omega'$ such that $f(p) = 0$ and the $m \times m$ matrix with entries $(D_{n + j}f^i(p))$ is invertible, there exist $A \subset \Omega$ and $B \subset \Omega'$ both open and $g : A \to B$ continuously differentiable such that
  $$\forall y \in B,\ (\exists x \in A : f(x, y) = 0) \iff (\exists x \in A : y = g(x))$$
\end{thm}

\subsubsection{* Equivalence of the two theorems}

Not examinable.

\section{Metric and topological spaces}

\subsection{Metric spaces}

\subsubsection{Motivation and definition}

\begin{defn}
	A \textbf{metric} on a set $X$ is a function $\d : X \times X \to \mathbb{R}$ where for all $x, y, z \in X$
	\begin{itemize}
		\item $\d(x, y) \geq 0$
		\item $\d(x, y) = 0 \iff x = y$
		\item $\d(x, y) = \d(y, x)$
		\item $\d(x, y) \leq \d(x, z) + \d(z, y)$
	\end{itemize}
\end{defn}

\begin{defn}
	A \textbf{metric space} is a pair $(X, \d)$ where $\d$ is a metric on $X$.
	The elements of $X$ are \textbf{points} and $\d$ is the \textbf{distance} between two points.
\end{defn}

\subsubsection{Examples of metric spaces}

\begin{defn*}
	A common metric on $\mathbb{R}$ is $\d_1(x, y) = \lvert x - y \rvert$.
\end{defn*}

\begin{defn*}
	The \textbf{Euclidean metric} on $\mathbb{R}^n$ is $\d_2(x, y) = \lvert\lvert x - y \rvert\rvert$.
\end{defn*}

\begin{lem}
	For $f : [a, b] \to \mathbb{R}$ continuous and non-negative, with $f(x)$ non-zero for at least one $x$, $\int_a^bf(t)\d t > 0$.
\end{lem}

\begin{defn*}
	For any set $X$, the \textbf{discrete metric} is 
	$$\d_{\text{disc}}(x, y) = \begin{cases}0 & \text{if } x = y\\1 & \text{if } x \neq y\end{cases}$$
\end{defn*}

\begin{defn*}
	The \textbf{British rail metric} on $\mathbb{R}$ is 
	$$\d_{\text{rail}}(x, y) = \begin{cases}\length{x - y} & \text{if } x = ky, k \in \mathbb{R}\\\length{x} + \length{y} & \text{otherwise}\end{cases}$$
\end{defn*}

\begin{defn*}
	The \textbf{supremum} or \textbf{uniform metric} on $C([a, b])$ is
	$$\d_{\infty}(f, g) = \max{\lvert f(t) - g(t) \rvert}$$
\end{defn*}

\begin{defn}
	For a metric space $(X, \d)$ and a subset $Y \subset X$, the restriction of $\d$ to $Y$ is the \textbf{induced metric} $\d\mid_Y$.
	$(Y, \d\mid_Y)$ is a \textbf{metric subspace} of $X$, and is also an MS.
\end{defn}

\begin{defn}
	For metric spaces $(X, \d_X)$ and $(Y, \d_Y)$ and a metric $\d$ on $X \times Y$ defined in terms of $\d_X$ and $\d_Y$, $(X \times Y, \d)$ is a \textbf{product metric space}.
	Important examples of such a $\d$ include:
	\begin{itemize}
		\item $\d((x_1, x_2), (y_1, y_2)) = \max{\{\d_1(x_1, y_1), \d_2(x_2, y_2)\}}$
		\item $\d((x_1, x_2), (y_1, y_2)) = \d_1(x_1, y_1) + \d_2(x_2, y_2)$
	\end{itemize}
\end{defn}

\subsubsection{Normed vector spaces}

\begin{defn}
	A \textbf{norm} on a vector space $V$ over $\mathbb{R}$ is a function $\length{\cdot} : V \to \mathbb{R}$ where for all $u, v \in V$ and $\lambda \in \mathbb{R}$
	\begin{itemize}
		\item $\length{v} \geq 0$
		\item $\length{v} = 0 \iff v = 0$
		\item $\length{\lambda v} = \lvert\lambda\rvert\length{v}$
		\item $\length{u + v} \leq \length{u} + \length{v}$
	\end{itemize}
\end{defn}

\begin{defn*}
	A \textbf{normed vector space} is a pair $(V, \length{\cdot})$ where $\length{\cdot}$ is a norm on $V$.
\end{defn*}

\begin{lem}
	For a normed vector space $(V, \length{\cdot})$, $\d_{\length{\ }}(u, v) = \length{u - v}$ is a metric on $V$.
\end{lem}

\subsubsection{Open sets in metric spaces}

\begin{defn}
	The \textbf{ball} of radius $\epsilon$ about $x$ (or \textbf{$\epsilon$-neighbourhood} of $x$) is
	$$B_{\epsilon}(x) = \{y \in X \mid d(x, y) < \epsilon\}$$
\end{defn}

\begin{defn}
	$U \subseteq X$ is \textbf{open} $\iff \forall u \in U,\  \exists \delta > 0 : B_{\delta}(u) \subseteq U$
\end{defn}

\begin{lem}
	All balls are open.
\end{lem}

\begin{lem}
	In any $(X, d)$, $X$ and $\emptyset$ are open.
\end{lem}

\begin{lem}
	A union of open sets is open.
\end{lem}

\begin{lem}
	An intersection of finitely many open sets is open.
\end{lem}

\begin{defn}
	Two metrics $\d_1$ and $\d_2$ on $X$ are \textbf{topologically equivalent} iff
	$$\forall U \subseteq X ,\  (U \text{ open in } (X, \d_1) \iff U \text{ open in } (X, \d_2))$$
\end{defn}

\subsubsection{Convergence in metric spaces}

\begin{defn}
	A sequence $(x_n)$ of elements of $X$ \textbf{converges} in $(X, \d)$ $\iff$ there exists a \textbf{limit} $x \in X$ such that
	$$\forall\epsilon > 0,\  \exists N \in \mathbb{N} : \forall n \geq N,\  \d(x_n, x) < \epsilon$$
	We use the usual notation of $x_n \to x$.
\end{defn}

\begin{defn*}
	A sequence $(x_n)$ is \textbf{eventually constant} iff
	$$\exists N \in \mathbb{N} : \forall n \geq N,\ x_n = x_N$$
\end{defn*}

\begin{prop*}
	Under the discrete metric, a sequence converges iff it is eventually constant.
\end{prop*}

\begin{lem}
	If the limit of a sequence exists, it is unique.
\end{lem}

\begin{cor}
	If $\d_1$ and $\d_2$ are topologically equivalent, convergence under $\d_1$ is equivalent to convergence under $\d_2$.
\end{cor}

\subsubsection{Closed sets in metric spaces}

\begin{defn}
	$V \subseteq X$ is \textbf{closed} in $(X, \d)$ $\iff$ all convergent sequences whose elements are all in $V$ have their limit also in $V$.
\end{defn}

\begin{prop*}
	Under the discrete metric, all sets are closed.
\end{prop*}

\begin{thm}
	$V$ is closed $\iff X \setminus V$ is open.
\end{thm}

\begin{lem}
	An intersection of closed sets is closed.
	A union of finitely many closed sets is closed.
\end{lem}

\subsubsection{Interior, isolated, limit, and boundary points in metric spaces}

\begin{defn}
	For $V \subset X$, $x \in X$ (not necessarily $x \in V$):
	\begin{itemize}
		\item $x$ is an \textbf{interior} or \textbf{inner point} of $V \iff \exists \delta > 0 : B_{\delta}(x) \subset V$ 
		\item $x$ is an \textbf{isolated point} of $V \iff \exists \delta > 0 : V \cap B_{\delta}(x) = \{x\}$
		\item $x$ is a \textbf{limit} or \textbf{accumulation point} of $V \iff \forall \delta > 0,\ (B_{\delta}(x) \cap V) \setminus \{x\} \neq \emptyset$
		\item $x$ is a \textbf{boundary point} of $V \iff \forall \delta > 0,\ B_{\delta}(x) \cap V \neq \emptyset$ and $B_{\delta}(x) \setminus V \neq \emptyset$
	\end{itemize}
\end{defn}

\begin{defn}
	For $(X, \d)$ and $V \subset X$:
	\begin{itemize}
		\item The \textbf{interior} of $V$ is $V^{\circ}$, the set of its interior points.
		\item The \textbf{closure} of $V$ is $\overline{V}$, the union of $V$ and the set of its limit points.
		\item The \textbf{boundary} of $V$ is $\partial V$, the set of its boundary points.
	\end{itemize}
\end{defn}

\begin{prop*}
	$V$ is open $\iff V = V^{\circ}$. 
\end{prop*}

\begin{prop*}
	$V$ is closed $\iff V = \overline{V}$.
\end{prop*}

\begin{prop*}
	$V \subseteq W \implies V^{\circ} \subseteq W^{\circ}$
\end{prop*}

\begin{prop*}
	$V \subseteq W \implies \overline{V} \subseteq \overline{W}$
\end{prop*}

\begin{prop*}
	$\overline{V \cup W} = \overline{V} \cup \overline{W}$
\end{prop*}

\begin{lem}
	$x$ is a limit point of $V \iff$ there is a sequence of points in $V \setminus \{x\}$ converging to $x$.
\end{lem}

\begin{defn}
	$V \subseteq X$ is \textbf{dense} in $X \iff \overline{V} = X$.
	$(X, \d)$ is \textbf{separable} $\iff$ there is a countable dense subset.
\end{defn}

\subsubsection{Continuous maps of metric spaces}

\begin{not*}
	From now on, let $(X, \d_X)$ and $(Y, \d_Y)$ be metric spaces and let $f : X \to Y$ be a map.
\end{not*}

\begin{defn}
	$f$ is \textbf{continuous at} $x \in X$ iff
	$$\forall \epsilon > 0,\ \exists \delta > 0 : \forall x' \in X,\ (\d_X(x, x') < \delta \implies \d_Y(f(x), f(x')) < \epsilon)$$
	$f$ is \textbf{continuous} $\iff f$ is continuous at every $x \in X$.
	$f$ is \textbf{uniformly continuous} $\iff f$ is continuous with $\delta$ independent of $x$ (but still dependent on $\epsilon$).
\end{defn}

\begin{thm}
	$f$ is continuous $\iff \forall U \subseteq Y$ \textnormal{(}$U$ is open $\implies f^{-1}(U)$ is open\textnormal{)}.
\end{thm}

\begin{prop*}
	$f$ is continuous $\iff \forall U \subseteq Y$ \textnormal{(}$U$ is closed $\implies f^{-1}(U)$ is closed\textnormal{)}
\end{prop*}

\begin{thm}
	$f$ is continuous at $x \in X \iff$ for all sequences $(x)_n$, $x_n \to x \implies f(x_n) \to f(x)$.
\end{thm}

\begin{defn}
	$f$ is a \textbf{homeomorphism} $\iff f$ is bijective and continuous and $f^{-1}$ is continuous.
	Two metric spaces are \textbf{homeomorphic} $\iff$ there is a homeomorphism between them.
\end{defn}

\begin{defn}
	$f$ is \textbf{Lipschitz} iff
	$$\exists M > 0 : \forall x_1, x_2 \in X,\ \d_Y(f(x_1), f(x_2)) \leq M\d_X(x_1, x_2)$$
	$f$ is \textbf{bi-Lipschitz} iff
	$$\exists M_1, M_2 > 0 : \forall x_1, x_2 \in X,\ M_2\d_X(x_1, x_2) \leq \d_Y(f(x_1), f(x_2)) \leq M_1\d_X(x_1, x_2)$$
	$f$ is an \textbf{isometry} or \textbf{distance preserving} iff
	$$\forall x_1, x_2 \in X,\ \d_Y(f(x_1), f(x_2)) = \d_X(x_1, x_2)$$
\end{defn}

\begin{prop*}
	$f$ is surjective and bi-Lipschitz $\implies f$ is a homeomorphism.
\end{prop*}

\subsection{Topological spaces}

\subsubsection{Motivation}

Nothing important in this section.

\subsubsection{Topology on a set}

\begin{defn}
	A \textbf{topology} on a set $A$ is $\tau$, a collection of subsets of $A$ where
	\begin{itemize}
		\item $\emptyset \in \tau$
		\item $A \in \tau$
		\item Any union of elements of $\tau$ is in $\tau$
		\item Any intersection of finitely many elements of $\tau$ is in $\tau$
	\end{itemize}
	A \textbf{topological space} is a pair $(A, \tau)$ where $\tau$ is a topology on $A$.
	The elements of $A$ are \textbf{points}.
	The elements of $\tau$ are \textbf{open sets}.
	An open set containing $a \in A$ is a \textbf{neighbourhood} of $a$.
\end{defn}

\begin{defn*}
	The \textbf{coarse topology} on any $A$ is $\{\emptyset, A\}$.
\end{defn*}

\begin{defn*}
	The \textbf{discrete topology} on any $A$ is the set of all its subsets.
\end{defn*}

\begin{defn*}
	The \textbf{Sierpinski topological space} is $(\{a, b\}, \{\emptyset, \{a, b\}, \{b\}\})$
\end{defn*}

\begin{defn*}
	The \textbf{order topology} on $\mathbb{R}$ is the collection of all sets of the form $(a, +\infty)$ with $a \in \mathbb{R} \cup \{-\infty, +\infty\}$, where $(+\infty, +\infty) = \emptyset$.
\end{defn*}

\begin{defn*}
	The \textbf{co-finite topology} on any $A$ is the collection of subsets whose complements are finite, and also the empty set.
\end{defn*}

\begin{defn*}
	The \textbf{induced topology} on any $X$ from a metric d on $X$ is the collection of open sets under d.
	Two topologically equivalent metrics induce the same topology.
\end{defn*}

\begin{defn*}
	The \textbf{Euclidean topology} is the topology induced from the Euclidean metric.
\end{defn*}

\begin{defn*}
	$(A, \tau)$ is \textbf{metrisable} if there is a metric on $A$ which induces $\tau$.
\end{defn*}

\begin{defn*}
	For $(A, \tau)$ and $Y \subset A$, the \textbf{induced} or \textbf{subspace topology} on $Y$ is $\tau_Y = \{U \cap Y \mid U \in \tau\}$
\end{defn*}

\begin{defn*}
	For $(A, \tau_A)$ and $(B, \tau_B)$, the \textbf{product topology} on $A \times B$ is
	$$\tau \star \mu = \{\Omega \subseteq A \times B \mid \forall (a, b) \in \Omega,\ \exists U \in \tau, V \in \mu : a \in U \land b \in V \land U \times V \in \Omega\}$$
\end{defn*}

\begin{defn}
	A topology $\tau$ is \textbf{stronger} or \textbf{finer} than another topology $\mu$ on the same set $\iff \mu \subset \tau$.
\end{defn}

\begin{prop*}
	The coarse topology is weaker than all others.
	The discrete topology is stronger than all others.
\end{prop*}

\begin{lem}
	$G \in \tau \iff \forall x \in G,\ \exists U \in \tau : x \in U \land U \subseteq G$.
\end{lem}

\begin{defn}
	For $\Omega \subseteq A$, $x \in \Omega$ is an \textbf{interior point} of $\Omega \iff \exists U \in \tau : z \in U \land U \subseteq \Omega$.
	The \textbf{interior} of $\Omega$ is $\Omega^{\circ}$, the set of all interior points.
\end{defn}

\subsubsection{Convergence, and the Hausdorff property}

\begin{defn}
	A sequence $(x_n)$ of elements of $A$ \textbf{converges} $\iff$ there is a \textbf{limit} $x$ such that
	$$\forall G \in \tau \text{ where } x \in \tau,\ \exists N \in \mathbb{N} : \forall n \geq N,\ x_n \in G$$.
	Limits are not necessarily unique.
\end{defn}

\begin{prop*}
	Under the coarse topology, all sequences are convergent.
	Under the discrete topology, $(x_n)$ converges $\iff (x_n)$ is eventually constant.
\end{prop*}

\begin{defn*}
	$U$ and $V$ \textbf{separate} $x$ and $y \iff x \in U \land y \in V \land U \cap V = \emptyset$.
\end{defn*}

\begin{defn}
	A topological space is \textbf{Hausdorff} $\iff$ any pair of distinct elements can be separated.
\end{defn}

\begin{thm}
	In Hausdorff topological spaces, limits are unique.
\end{thm}

\subsubsection{Closed sets in topological spaces}

\begin{defn}
	$V \subseteq A$ is \textbf{closed} in $(A, \tau) \iff A \setminus V$ is open in $(A, \tau)$.
\end{defn}

\begin{thm}
	The lecturer messed up the numbering.
\end{thm}

\begin{lem}
	$\emptyset$ and $A$ are closed.
	An intersection of closed sets is closed.
	A union of finitely many closed sets is closed.
\end{lem}

\begin{lem}
	For $(A, \tau)$ Hausdorff and $a \in A$, $\{a\}$ is closed.
\end{lem}

\begin{defn}
	For $\Omega \subseteq A$, $x \in \Omega$ is a \textbf{limit} or \textbf{accumulation point} of $\Omega$ iff 
	$$\forall U \in \tau,\ x \in U \implies (X \cup U) \setminus \{x\} \neq \emptyset$$
	The \textbf{closure} of $\Omega$ is $\overline{\Omega}$, the union of $\Omega$ and the set of its limit points.
\end{defn}

\begin{lem}
	$S \subset T \implies \overline{S} \subset \overline{T}$. $S$ is closed $\iff S = \overline{S}$.
\end{lem}

\subsubsection{Continuous maps on topological spaces}

\begin{not*}
	In this section, let $(A, \tau_A)$ and $(B, \tau_B)$ be topological spaces and let $f : A \to B$ be a map.
\end{not*}

\begin{defn}
	$f$ is \textbf{continuous} $\iff \forall U \in \tau_B,\ f^{-1}(U) \in \tau_A$.
\end{defn}

\begin{prop*}
	$\tau_A$ is the discrete topology $\implies f$ is continuous.
	$\tau_B$ is the coarse topology $\implies f$ is continuous.
\end{prop*}

\begin{thm}
	$f$ is continuous $\iff \forall U \in \tau_B,\ U$ is closed $\implies f^{-1}(U)$ is closed. 
\end{thm}

\begin{thm}
	A composition of continuous maps is continuous.
\end{thm}

\begin{lem}
	The constant map is continuous.
\end{lem}

\begin{defn}
	$f$ is a \textbf{homeomorphism} $\iff f$ is bijective and continuous and $f^{-1}$ is continuous.
	Two topological spaces are \textbf{topologically equivalent} or \textbf{homeomorphic} $\iff$ there is a homeomorphism between them.
\end{defn}

\begin{prop*}
	If $(A, \tau_A)$ and $(B, \tau_B)$ are homeomorphic, $(A, \tau_A)$ is Hausdorff $\iff (B, \tau_B)$ is Hausdorff.
\end{prop*}

\subsection{Connectedness}

\subsubsection{Connected sets}

\begin{not*}
	From now on $(X, \d)$, $(X, \d_X)$ and $(Y, \d_Y)$.
\end{not*}

\begin{defn}
	$T \subseteq X$ is \textbf{disconnected} if there are open sets $U, V \subseteq X$ such that
	\begin{itemize}
		\item $U \cap V = \emptyset$
		\item $T \subseteq U \cup V$
		\item $T \cap U \neq \emptyset$ and $T \cap V \neq \emptyset$
	\end{itemize}
\end{defn}

\begin{defn}
	$T$ is \textbf{connected} $\iff T$ is not disconnected.
\end{defn}

\begin{prop*}
	$X$ is connected $\iff$ the only subsets which are both open and closed are $X$ and $\emptyset$.
\end{prop*}

\begin{lem}
	$T$ is disconnected $\iff$ there is a continuous map $f : T \to \mathbb{R}$ with $f(T) = \{0, 1\}$.
\end{lem}

\begin{lem}
	$S \subseteq \mathbb{R}$ is an interval $\iff \forall x, y \in S, z \in \mathbb{R},\ x < z < y \implies z \in S$.
\end{lem}

\begin{thm}
	In $(\mathbb{R}, \d_1)$, all connected sets are intervals.
\end{thm}

\begin{thm}
	In $(\mathbb{R}, \d_1)$, all closed intervals (of the form $[a, b]$) are connected.
\end{thm}

\begin{prop*}
	In fact, all intervals in $(\mathbb{R}, \d_1)$ are connected.
\end{prop*}

\subsubsection{Continuous maps and connected sets}

\begin{thm}
	If $f : X \to Y$ is continuous and $S \subseteq X$ is connected, $f(S)$ is connected.
\end{thm}

\begin{cor}
	If $(X, \d_X)$ and $(Y, \d_Y)$ are homeomorphic, $X$ is connected $\iff Y$ is connected.
\end{cor}

\begin{thm}
	If $X$ is connected, $f : X \to \mathbb{R}$ is continuous, and $a, b \in X$
	$$f(a) < 0 \land f(b) > 0 \implies \exists c \in X : f(c) = 0$$
\end{thm}

\begin{cor}
	If $f : [a, b] \to \mathbb{R}$ is continuous and $x, y \in [a, b]$
	$$f(x) < 0 \land f(y) > 0 \implies \exists z \in [a, b] : f(z) = 0$$
\end{cor}

\subsubsection{Path connected sets}

\begin{defn}
	A \textbf{path} from $a$ to $b$ is a continuous map $f : [0, 1] \to X$ where $f(0) = a$ and $f(1) = b$.
\end{defn}

\begin{defn}
	$(X, \d)$ is \textbf{path-connected} if every pair of points is joined by a path.
\end{defn}

\begin{thm}
	$(X, \d)$ is path connected $\implies (X, \d)$ is connected.
\end{thm}

\subsection{Compactness}

\subsubsection{Compactness by covers}

\begin{defn}
	An \textbf{open cover} for $Y \subseteq X$ is a collection $\mathcal{R}$ of open subsets of $X$ where $Y \subseteq \bigcup_{U \in \mathcal{R}}U$.
	A \textbf{sub-cover} of $\mathcal{R}$ for $Y$ is $\mathcal{C} \subseteq \mathcal{R}$ such that $\mathcal{C}$ is also an open cover for $Y$.
	A \textbf{finite cover} is an open cover with finitely many elements.
\end{defn}

\begin{defn}
	$Y$ is \textbf{compact} $\iff$ every open cover for $Y$ has a finite sub-cover.
\end{defn}

\begin{prop}
	In $(\mathbb{R}, \d_1)$, closed intervals are compact.
\end{prop}

\begin{prop}
	$X$ is compact and $Y \subseteq X$ is closed $\implies Y$ is compact.
\end{prop}

\begin{thm}
	$Y$ is compact $\implies Y$ is closed.
\end{thm}

\begin{thm}
	For both of the examples of product metric spaces in Definition 2.4, $X$ and $Y$ are compact $\implies X \times Y$ is compact.
\end{thm}

\begin{cor}
	A Cartesian product of closed intervals is compact in $\mathbb{R}^n$.
\end{cor}

\begin{defn}
	$Z \subseteq X$ is \textbf{bounded} iff
	$$\exists M \in \mathbb{R} : \forall x, y \in Z,\ \d(x, y) \leq M$$
	For an arbitrary set $S$, $f : S \to X$ is \textbf{bounded} $\iff f(S)$ is bounded.
\end{defn}

\begin{prop*}
	A union of finitely many bounded sets is bounded.
\end{prop*}

\begin{lem}
	$(X, \d)$ is compact $\implies X$ is bounded.
\end{lem}

\begin{thm}
	(Heine Borel Theorem)
	In $(\mathbb{R}^n, \d_2)$, a subset is compact $\iff$ it is closed and bounded.
\end{thm}

\subsubsection{Sequential compactness}

\begin{defn}
	$(X, \d)$ is \textbf{sequentially compact} $\iff$ every sequence has a convergent subsequence.
\end{defn}

\begin{lem}
	$(x_n)$ has a convergent subsequence $\iff \exists x \in X : \forall \epsilon > 0$, there are infinitely many $i$ where $x_i \in B_{\epsilon}(x)$.
\end{lem}

\begin{prop*}
	$(X, \d)$ is sequentially compact $\implies X$ is bounded.
\end{prop*}

\begin{thm}
	$(X, \d)$ is compact $\implies (X, \d)$ is sequentially compact.
\end{thm}

\begin{thm}
	(Bolzano-Weierstrass Theorem)
	In $\mathbb{R}^n$, every bounded sequence has a convergent subsequence.
\end{thm}

\begin{thm}
	$(X, \d)$ is sequentially compact $\implies (X, \d)$ is compact.
\end{thm}

\subsubsection{Continuous maps and compact sets}

\begin{thm}
	$Z \subseteq X$ is compact and $f : X \to Y$ is continuous $\implies f(Z)$ is compact.
\end{thm}

\begin{cor}
	If $(X, \d_X)$ and $(Y, \d_Y)$ are homeomorphic, $X$ is compact $\iff Y$ is compact.
\end{cor}

\begin{thm}
	$X$ is compact and $f : X \to Y$ is continuous $\implies f$ is uniformly continuous.
\end{thm}

\begin{cor}
	$f : [a, b] \to \mathbb{R}$ is continuous $\implies f$ is uniformly continuous.
\end{cor}

\begin{thm}
	If $X$ is compact and $f : X \to \mathbb{R}$ is continuous, $f$ is bounded from above and below and attains its upper and lower bounds.
\end{thm}

\begin{thm}
	For $f : \mathbb{R} \to \mathbb{R}$ continuous and $[a, b] \subseteq \mathbb{R}$, $f([a, b])$ is a closed interval.
\end{thm}

\subsection{Completeness}

\subsubsection{Complete metric spaces and Banach space}

\begin{defn}
	$(x_n)$ is \textbf{Cauchy} iff
	$$\forall \epsilon > 0,\ \exists N \in \mathbb{N} : \forall n, m > N,\ \d(x_n, x_m) < \epsilon$$
\end{defn}

\begin{prop*}
	$(x_n)$ is convergent $\implies (x_n)$ is Cauchy.
\end{prop*}

\begin{prop*}
	If a Cauchy sequence has a subsequence converging to some $x$, the whole sequence convergence to $x$.
\end{prop*}

\begin{defn}
	$(X, \d)$ is \textbf{complete} $\iff$ all Cauchy sequences in $X$ converge.
	$(V, \length{\cdot})$ is a \textbf{Banach space} $\iff (V, \d_{\length{\ }})$ is complete.
\end{defn}

\begin{lem}
	For all $n$, $(\mathbb{R}^n, \d_2)$ is complete.
\end{lem}

\begin{prop}
	The metric space $(C([a, b]), \d_2)$, where
	$$\d_2(f, g) = \sqrt{\int_a^b\lvert f(t) - g(t) \rvert^2dt}$$
	is not complete.
\end{prop}

\begin{thm}
	If $(f_n)$ is a sequence of continuous functions which converges uniformly, its limit is also continuous.
\end{thm}

\begin{thm}
	The metric space $(C([a, b]), \d_{\infty})$, where
	$$\d_{\infty}(f, g) = \sup_{t \in [a, b]} \lvert f(t) - g(t) \rvert$$
	is complete.
\end{thm}

\begin{thm}
	$(X, \d)$ is compact $\implies (X, \d)$ is complete.
\end{thm}

\subsubsection{Arzel\`a-Ascoli}

\begin{defn}
	A collection $\mathcal{C}$ of functions $f : [a, b] \to \mathbb{R}$ is \textbf{uniformly bounded} iff
	$$\exists M > 0 : \forall f \in \mathcal{C}, x \in [a, b],\ \lvert f(x)\rvert < M$$
	$\mathcal{C}$ is \textbf{uniformly equi-continuous} iff
	$$\forall \epsilon > 0,\ \exists \delta > 0 : \forall f \in \mathcal{C}, x, y \in [a, b],\ \lvert x - y \rvert < \delta \implies \lvert f(x) - f(y) \rvert < \epsilon$$
\end{defn}

\begin{thm}
	(Arzel\`a-Ascoli Theorem)
	If $\mathcal{C}$ is a uniformly bounded and equi-continuous collection of continuous functions, every sequence in $\mathcal{C}$ has a convergent subsequence in $(C([a, b]), \d_{\infty})$ with $\d_{\infty}$ defined as in Theorem 2.52.
\end{thm}

\subsubsection{Fixed point theorem}

\begin{defn}
	$f : X \to Y$ is \textbf{contracting} iff
	$$\exists K \in (0, 1) : \forall a, b \in X,\ \d_Y(f(a), f(b)) \leq K\d_X(a, b)$$
\end{defn}

\begin{thm}
	(Banach Fixed Point Theorem)
	If $(X, \d)$ is non-empty and complete and $f : X \to X$ is contracting, $f$ has a unique fixed point.
\end{thm}

\end{document}
